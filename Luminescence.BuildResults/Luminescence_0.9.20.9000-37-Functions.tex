\begin{table}[ht]
\centering
\begin{tabular}{rllllllll}
  \hline
 & Name & Title & Description & Version & m.Date & m.Time & Author & Citation \\ 
  \hline
1 & analyse\_Al2O3C\_CrossTalk & Al2O3:C Reader Cross Talk Analysis & The function provides the analysis of cross-talk measurements on a FI lexsyg SMART reader using Al2O3:C chips & 0.1.2
 &  &  & Sebastian Kreutzer, Institute of Geography, Heidelberg University (Germany)$<$br /$>$ , RLum Developer Team &  \\ 
  2 & analyse\_Al2O3C\_ITC & Al2O3 Irradiation Time Correction Analysis & The function provides a very particular analysis to correct the irradiation time while irradiating Al2O3:C chips in a luminescence reader. & 0.1.1
 &  &  & Sebastian Kreutzer, Institute of Geography, Heidelberg University (Germany)$<$br /$>$ , RLum Developer Team &  \\ 
  3 & analyse\_Al2O3C\_Measurement & Al2O3:C Passive Dosimeter Measurement Analysis & The function provides the analysis routines for measurements on a FI lexsyg SMART reader using Al2O3:C chips according to Kreutzer et al., 2018 & 0.2.6
 &  &  & Sebastian Kreutzer, Institute of Geography, Heidelberg University (Germany)$<$br /$>$ , RLum Developer Team &  \\ 
  4 & analyse\_baSAR & Bayesian models (baSAR) applied on luminescence data & This function allows the application of Bayesian models on luminescence data, measured with the single-aliquot regenerative-dose (SAR, Murray and Wintle, 2000) protocol. In particular, it follows the idea proposed by Combès et al., 2015 of using an hierarchical model for estimating a central equivalent dose from a set of luminescence measurements. This function is (I) the adoption of this approach for the R environment and (II) an extension and a technical refinement of the published code. & 0.1.33
 &  &  & Norbert Mercier, IRAMAT-CRP2A, Université Bordeaux Montaigne (France)  $<$br /$>$ Sebastian Kreutzer, Institute of Geography, Heidelberg University (Germany)  $<$br /$>$ The underlying Bayesian model based on a contribution by Combès et al., 2015.$<$br /$>$ , RLum Developer Team &  \\ 
  5 & analyse\_FadingMeasurement & Analyse fading measurements and returns the fading rate per decade (g-value) & The function analysis fading measurements and returns a fading rate including an error estimation. The function is not limited to standard fading measurements, as can be seen, e.g., Huntley and Lamothe (2001). Additionally, the density of recombination centres (rho') is estimated after Kars et al. (2008). & 0.1.21
 &  &  & Sebastian Kreutzer, Institute of Geography, Heidelberg University (Germany)  $<$br /$>$ Christoph Burow, University of Cologne (Germany)$<$br /$>$ , RLum Developer Team &  \\ 
  6 & analyse\_IRSAR.RF & Analyse IRSAR RF measurements & Function to analyse IRSAR RF measurements on K-feldspar samples, performed using the protocol according to Erfurt et al. (2003) and beyond. & 0.7.8
 &  &  & Sebastian Kreutzer, Institute of Geography, Heidelberg University (Germany)$<$br /$>$ , RLum Developer Team &  \\ 
  7 & analyse\_pIRIRSequence & Analyse post-IR IRSL measurement sequences & The function performs an analysis of post-IR IRSL sequences including curve fitting on  RLum.Analysis  objects. & 0.2.4
 &  &  & Sebastian Kreutzer, Institute of Geography, Heidelberg University (Germany)$<$br /$>$ , RLum Developer Team &  \\ 
  8 & analyse\_portableOSL & Analyse portable CW-OSL measurements & The function analyses CW-OSL curve data produced by a SUERC portable OSL reader and produces a combined plot of OSL/IRSL signal intensities, OSL/IRSL depletion ratios and the IRSL/OSL ratio. & 0.0.3
 &  &  & Christoph Burow, University of Cologne (Germany)$<$br /$>$ , RLum Developer Team &  \\ 
  9 & analyse\_SAR.CWOSL & Analyse SAR CW-OSL measurements & The function performs a SAR CW-OSL analysis on an RLum.Analysis  object including growth curve fitting. & 0.9.14
 &  &  & Sebastian Kreutzer, Geography \& Earth Sciences, Aberystwyth University$<$br /$>$ (United Kingdom)$<$br /$>$ , RLum Developer Team &  \\ 
  10 & Analyse\_SAR.OSLdata & Analyse SAR CW-OSL measurements. & The function analyses SAR CW-OSL curve data and provides a summary of the measured data for every position. The output of the function is optimised for SAR OSL measurements on quartz. & 0.2.17
 &  &  & Sebastian Kreutzer, Institute of Geography, Heidelberg University (Germany) $<$br /$>$ Margret C. Fuchs, HZDR, Freiberg (Germany)$<$br /$>$ , RLum Developer Team &  \\ 
  11 & analyse\_SAR.TL & Analyse SAR TL measurements & The function performs a SAR TL analysis on a RLum.Analysis  object including growth curve fitting. & 0.3.0
 &  &  & Sebastian Kreutzer, Institute of Geography, Heidelberg University (Germany)$<$br /$>$ , RLum Developer Team &  \\ 
  12 & apply\_CosmicRayRemoval & Function to remove cosmic rays from an RLum.Data.Spectrum S4 class object & The function provides several methods for cosmic-ray removal and spectrum smoothing  RLum.Data.Spectrum  objects and such objects embedded in  list  or RLum.Analysis  objects. & 0.3.0
 &  &  & Sebastian Kreutzer, Institute of Geography, Heidelberg University (Germany)$<$br /$>$ , RLum Developer Team &  \\ 
  13 & apply\_EfficiencyCorrection & Function to apply spectral efficiency correction to RLum.Data.Spectrum S4$<$br /$>$ class objects & The function allows spectral efficiency corrections for RLum.Data.Spectrum S4 class objects & 0.2.0
 &  &  & Sebastian Kreutzer, IRAMAT-CRP2A, UMR 5060, CNRS-Université Bordeaux Montaigne (France) $<$br /$>$ Johannes Friedrich, University of Bayreuth (Germany)$<$br /$>$ , RLum Developer Team &  \\ 
  14 & as & as() - RLum-object coercion & for  [RLum.Analysis-class]   for  [RLum.Data.Curve-class]   for  [RLum.Data.Image-class]   for  [RLum.Data.Spectrum-class]   for  [RLum.Results-class] &  &  &  &  &  \\ 
  15 & BaseDataSet.ConversionFactors & Base data set of dose-rate conversion factors & Collection of published dose-rate conversion factors to convert concentrations of radioactive isotopes to dose rate values. &  &  &  &  &  \\ 
  16 & BaseDataSet.CosmicDoseRate & Base data set for cosmic dose rate calculation & Collection of data from various sources needed for cosmic dose rate calculation &  &  &  &  &  \\ 
  17 & BaseDataSet.FractionalGammaDose & Base data set of fractional gamma-dose values & Collection of (un-)published fractional gamma dose-rate values to scale the gamma-dose rate considering layer-to-layer variations in soil radioactivity. &  &  &  &  &  \\ 
  18 & BaseDataSet.GrainSizeAttenuation & Base dataset for grain size attenuation data by Guérin et al. (2012) & Grain size correction data for beta-dose rates published by Guérin et al. (2012).  \#' @format  A  data.frame  seven columns and sixteen rows. Column headers are  GrainSize ,  Q\_K ,  FS\_K ,  Q\_Th ,  FS\_Th ,  Q\_U ,  FS\_U . Grain sizes are quoted in µm (e.g., 20, 40, 60 etc.) &  &  &  &  &  \\ 
  19 & bin\_RLum.Data & Channel binning - method dispatcher & Function calls the object-specific bin functions for RLum.Data S4 class objects. & 0.2.0
 &  &  & Sebastian Kreutzer, Institute of Geography, Heidelberg University (Germany)$<$br /$>$ , RLum Developer Team &  \\ 
  20 & calc\_AliquotSize & Estimate the amount of grains on an aliquot & Estimate the number of grains on an aliquot. Alternatively, the packing density of an aliquot is computed. & 0.31
 &  &  & Christoph Burow, University of Cologne (Germany)$<$br /$>$ , RLum Developer Team &  \\ 
  21 & calc\_AverageDose & Calculate the Average Dose and the dose rate dispersion & This functions calculates the Average Dose and their extrinsic dispersion and estimates the standard errors by bootstrapping based on the Average Dose Model by Guerin et al., 2017 & 0.1.5
 &  &  & Claire Christophe, IRAMAT-CRP2A, Université de Nantes (France),$<$br /$>$ Anne Philippe, Université de Nantes, (France),$<$br /$>$ Guillaume Guérin, IRAMAT-CRP2A, Université Bordeaux Montaigne, (France),$<$br /$>$ Sebastian Kreutzer, Institute of Geography, Heidelberg University (Germany)$<$br /$>$ , RLum Developer Team &  \\ 
  22 & calc\_CentralDose & Apply the central age model (CAM) after Galbraith et al. (1999) to a given$<$br /$>$ De distribution & This function calculates the central dose and dispersion of the De distribution, their standard errors and the profile log likelihood function for sigma. & 1.4.1
 &  &  & Christoph Burow, University of Cologne (Germany)  $<$br /$>$ Based on a rewritten S script of Rex Galbraith, 2010$<$br /$>$ , RLum Developer Team &  \\ 
  23 & calc\_CobbleDoseRate & Calculate dose rate of slices in a spherical cobble & Calculates the dose rate profile through the cobble based on Riedesel and Autzen (2020).  Corrects the beta dose rate in the cobble for the grain size following results of Guérin et al. (2012). Sediment beta and gamma dose rates are corrected for the water content of the sediment using the correction factors of Aitken (1985). Water content in the cobble is assumed to be 0. & 0.1.0
 &  &  & Svenja Riedesel, Aberystwyth University (United Kingdom)  $<$br /$>$ Martin Autzen, DTU NUTECH Center for Nuclear Technologies (Denmark)$<$br /$>$ , RLum Developer Team &  \\ 
  24 & calc\_CommonDose & Apply the (un-)logged common age model after Galbraith et al. (1999) to a$<$br /$>$ given De distribution & Function to calculate the common dose of a De distribution. & 0.1.1
 &  &  & Christoph Burow, University of Cologne (Germany)$<$br /$>$ , RLum Developer Team &  \\ 
  25 & calc\_CosmicDoseRate & Calculate the cosmic dose rate & This function calculates the cosmic dose rate taking into account the soft- and hard-component of the cosmic ray flux and allows corrections for geomagnetic latitude, altitude above sea-level and geomagnetic field changes. & 0.5.2
 &  &  & Christoph Burow, University of Cologne (Germany)$<$br /$>$ , RLum Developer Team &  \\ 
  26 & calc\_FadingCorr & Apply a fading correction according to Huntley \& Lamothe (2001) for a given$<$br /$>$ g-value and a given tc & This function solves the equation used for correcting the fading affected age including the error for a given g-value according to Huntley \& Lamothe (2001). & 0.4.3
 &  &  & Sebastian Kreutzer, Institute of Geography, Heidelberg University (Germany)$<$br /$>$ , RLum Developer Team &  \\ 
  27 & calc\_FastRatio & Calculate the Fast Ratio for CW-OSL curves & Function to calculate the fast ratio of quartz CW-OSL single grain or single aliquot curves after Durcan \& Duller (2011). & 0.1.1
 &  &  & Georgina E. King, University of Bern (Switzerland)  $<$br /$>$ Julie A. Durcan, University of Oxford (United Kingdom)  $<$br /$>$ Christoph Burow, University of Cologne (Germany)$<$br /$>$ , RLum Developer Team &  \\ 
  28 & calc\_FiniteMixture & Apply the finite mixture model (FMM) after Galbraith (2005) to a given De$<$br /$>$ distribution & This function fits a k-component mixture to a De distribution with differing known standard errors. Parameters (doses and mixing proportions) are estimated by maximum likelihood assuming that the log dose estimates are from a mixture of normal distributions. & 0.4.2
 &  &  & Christoph Burow, University of Cologne (Germany)  $<$br /$>$ Based on a rewritten S script of Rex Galbraith, 2006.$<$br /$>$ , RLum Developer Team &  \\ 
  29 & calc\_FuchsLang2001 & Apply the model after Fuchs \& Lang (2001) to a given De distribution. & This function applies the method according to Fuchs \& Lang (2001) for heterogeneously bleached samples with a given coefficient of variation threshold. & 0.4.1
 &  &  & Sebastian Kreutzer, Institute of Geography, Heidelberg University (Germany)  $<$br /$>$ Christoph Burow, University of Cologne (Germany)$<$br /$>$ , RLum Developer Team &  \\ 
  30 & calc\_gSGC\_feldspar & Calculate Global Standardised Growth Curve (gSGC) for Feldspar MET-pIRIR & Implementation of the gSGC approach for feldspar MET-pIRIR by Li et al. (2015) & 0.1.0
 &  &  & Harrison Gray, USGS (United States),$<$br /$>$ Sebastian Kreutzer, Institute of Geography, Heidelberg University (Germany)$<$br /$>$ , RLum Developer Team &  \\ 
  31 & calc\_gSGC & Calculate De value based on the gSGC by Li et al., 2015 & Function returns De value and De value error using the global standardised growth curve (gSGC) assumption proposed by Li et al., 2015 for OSL dating of sedimentary quartz & 0.1.1
 &  &  & Sebastian Kreutzer, Institute of Geography, Heidelberg University (Germany)$<$br /$>$ , RLum Developer Team &  \\ 
  32 & calc\_HomogeneityTest & Apply a simple homogeneity test after Galbraith (2003) & A simple homogeneity test for De estimates & 0.3.0
 &  &  & Christoph Burow, University of Cologne (Germany), Sebastian Kreutzer,$<$br /$>$ IRAMAT-CRP2A, Université Bordeaux Montaigne (France)$<$br /$>$ , RLum Developer Team &  \\ 
  33 & calc\_Huntley2006 & Apply the Huntley (2006) model & A function to calculate the expected sample specific fraction of saturation based on the model of Huntley (2006) using the approach as implemented in Kars et al. (2008) or Guralnik et al. (2015). & 0.4.1
 &  &  & Georgina E. King, University of Bern (Switzerland)  $<$br /$>$ Christoph Burow, University of Cologne (Germany)$<$br /$>$ , RLum Developer Team &  \\ 
  34 & calc\_IEU & Apply the internal-external-uncertainty (IEU) model after Thomsen et al.$<$br /$>$ (2007) to a given De distribution & Function to calculate the IEU De for a De data set. & 0.1.1
 &  &  & Rachel Smedley, Geography \& Earth Sciences, Aberystwyth University (United Kingdom)  $<$br /$>$ Based on an excel spreadsheet and accompanying macro written by Kristina Thomsen.$<$br /$>$ , RLum Developer Team &  \\ 
  35 & calc\_Kars2008 & Apply the Kars et al. (2008) model (deprecated) & A function to calculate the expected sample specific fraction of saturation following Kars et al. (2008) and Huntley (2006). This function is deprecated and will eventually be removed. Please use  calc\_Huntley2006()  instead. & 0.4.0
 &  &  & Georgina E. King, University of Bern (Switzerland)  $<$br /$>$ Christoph Burow, University of Cologne (Germany)$<$br /$>$ , RLum Developer Team &  \\ 
  36 & calc\_Lamothe2003 & Apply fading correction after Lamothe et al., 2003 & This function applies the fading correction for the prediction of long-term fading as suggested by Lamothe et al., 2003. The function basically adjusts the \$L\_n/T\_n\$ values and fit a new dose-response curve using the function  plot\_GrowthCurve . & 0.1.0
 &  &  & Sebastian Kreutzer, Institute of Geography, Heidelberg University (Germany), Norbert Mercier,$<$br /$>$ IRAMAT-CRP2A, Université Bordeaux Montaigne (France)$<$br /$>$ , RLum Developer Team &  \\ 
  37 & calc\_MaxDose & Apply the maximum age model to a given De distribution & Function to fit the maximum age model to De data. This is a wrapper function that calls  calc\_MinDose  and applies a similar approach as described in Olley et al. (2006). & 0.3.1
 &  &  & Christoph Burow, University of Cologne (Germany)  $<$br /$>$ Based on a rewritten S script of Rex Galbraith, 2010$<$br /$>$ , RLum Developer Team &  \\ 
  38 & calc\_MinDose & Apply the (un-)logged minimum age model (MAM) after Galbraith et al. (1999)$<$br /$>$ to a given De distribution & Function to fit the (un-)logged three or four parameter minimum dose model (MAM-3/4) to De data. & 0.4.4
 &  &  & Christoph Burow, University of Cologne (Germany)  $<$br /$>$ Based on a rewritten S script of Rex Galbraith, 2010  $<$br /$>$ The bootstrap approach is based on a rewritten MATLAB script of Alastair Cunningham.  $<$br /$>$ Alastair Cunningham is thanked for his help in implementing and cross-checking the code.$<$br /$>$ , RLum Developer Team &  \\ 
  39 & calc\_OSLLxTxDecomposed & Calculate Lx/Tx ratio for decomposed CW-OSL signal components & Calculate  Lx/Tx  ratios from a given set of decomposed CW-OSL curves decomposed by  [OSLdecomposition::RLum.OSL\_decomposition] & 0.1.0
 &  &  & Dirk Mittelstrass$<$br /$>$ , RLum Developer Team &  \\ 
  40 & calc\_OSLLxTxRatio & Calculate  Lx/Tx  ratio for CW-OSL curves & Calculate  Lx/Tx  ratios from a given set of CW-OSL curves assuming late light background subtraction. & 0.8.0
 &  &  & Sebastian Kreutzer, Institute of Geography, Heidelberg University (Germany)$<$br /$>$ , RLum Developer Team &  \\ 
  41 & calc\_SourceDoseRate & Calculation of the source dose rate via the date of measurement & Calculating the dose rate of the irradiation source via the date of measurement based on: source calibration date, source dose rate, dose rate error. The function returns a data.frame that provides the input argument dose\_rate for the function  Second2Gray . & 0.3.2
 &  &  & Margret C. Fuchs, HZDR, Helmholtz-Institute Freiberg for Resource Technology (Germany)  $<$br /$>$ Sebastian Kreutzer, Institute of Geography, Heidelberg University (Germany)$<$br /$>$ , RLum Developer Team &  \\ 
  42 & calc\_Statistics & Function to calculate statistic measures & This function calculates a number of descriptive statistics for estimates with a given standard error (SE), most fundamentally using error-weighted approaches. & 0.1.7
 &  &  & Michael Dietze, GFZ Potsdam (Germany)$<$br /$>$ , RLum Developer Team &  \\ 
  43 & calc\_ThermalLifetime & Calculates the Thermal Lifetime using the Arrhenius equation & The function calculates the thermal lifetime of charges for given E (in eV), s (in 1/s) and T (in deg. C.) parameters. The function can be used in two operational modes: & 0.1.0
 &  &  & Sebastian Kreutzer, Institute of Geography, Heidelberg University (Germany)$<$br /$>$ , RLum Developer Team &  \\ 
  44 & calc\_TLLxTxRatio & Calculate the Lx/Tx ratio for a given set of TL curves -beta version- & Calculate Lx/Tx ratio for a given set of TL curves. & 0.3.3
 &  &  & Sebastian Kreutzer, Institute of Geography, Heidelberg University (Germany)  $<$br /$>$ Christoph Schmidt, University of Bayreuth (Germany)$<$br /$>$ , RLum Developer Team &  \\ 
  45 & calc\_WodaFuchs2008 & Obtain the equivalent dose using the approach by Woda and Fuchs 2008 & The function generates a histogram-like reorganisation of the data, to assess counts per bin. The log-transformed counts per bin are used to calculate the second derivative of the data (i.e., the curvature of the curve) and to find the central value of the bin hosting the distribution maximum. A normal distribution model is fitted to the counts per bin data to estimate the dose distribution parameters. The uncertainty of the model is estimated based on all input equivalent doses smaller that of the modelled central value. & 0.2.0
 &  &  & Sebastian Kreutzer, Institute of Geography, Heidelberg University (Germany), $<$br /$>$ Michael Dietze, GFZ Potsdam (Germany)$<$br /$>$ , RLum Developer Team &  \\ 
  46 & combine\_De\_Dr & Combine Dose Rate and Equivalent Dose Distribution & A Bayesian statistical analysis of OSL age requiring dose rate sample. Estimation contains a preliminary step for detecting outliers in the equivalent dose sample. & 0.1.0
 &  &  & Anne Philippe, Université de Nantes (France),$<$br /$>$ Jean-Michel Galharret, Université de Nantes (France),$<$br /$>$ Norbert Mercier, IRAMAT-CRP2A, Université Bordeaux Montaigne (France),$<$br /$>$ Sebastian Kreutzer, Institute of Geography, Heidelberg University (Germany)$<$br /$>$ , RLum Developer Team &  \\ 
  47 & convert\_Activity2Concentration & Convert Nuclide Activities to Concentrations and Vice Versa & The function performs the conversion of the specific activities into concentrations and vice versa for the radioelements U, Th, and K to harmonise the measurement unit with the required data input unit of potential analytical tools for, e.g. dose rate calculation or related functions such as  use\_DRAC . & 0.1.1
 &  &  & Margret C. Fuchs, Helmholtz-Institute Freiberg for Resource Technology (Germany)$<$br /$>$ , RLum Developer Team &  \\ 
  48 & convert\_BIN2CSV & Export Risoe BIN-file(s) to CSV-files & This function is a wrapper function around the functions  read\_BIN2R  and write\_RLum2CSV  and it imports a Risoe BIN-file and directly exports its content to CSV-files. If nothing is set for the argument  path  ( write\_RLum2CSV ) the input folder will become the output folder. & 0.1.0
 &  &  & Sebastian Kreutzer, Institute of Geography, Heidelberg University (Germany)$<$br /$>$ , RLum Developer Team &  \\ 
  49 & convert\_Concentration2DoseRate & Dose-rate conversion function & This function converts radionuclide concentrations (K in \%, Th and U in ppm) into dose rates (Gy/ka). Beta-dose rates are also attenuated for the grain size. Beta and gamma-dose rates are corrected for the water content. This function converts concentrations into dose rates (Gy/ka) and corrects for grain size attenuation and water content  Dose rate conversion factors can be chosen from Adamiec and Aitken (1998), Guerin et al. (2011), Liritzis et al. (201) and Cresswell et al. (2018). Default is Guerin et al. (2011).  Grain size correction for beta dose rates is achieved using the correction factors published by Guérin et al. (2012).  Water content correction is based on factors provided by Aitken (1985), with the factor for beta dose rate being 1.25 and for gamma 1.14. & 0.1.0
 &  &  & Svenja Riedesel, Aberystwyth University (United Kingdom)  $<$br /$>$ Martin Autzen, DTU NUTECH Center for Nuclear Technologies (Denmark)$<$br /$>$ , RLum Developer Team &  \\ 
  50 & convert\_Daybreak2CSV & Export measurement data produced by a Daybreak luminescence reader to CSV-files & This function is a wrapper function around the functions  read\_Daybreak2R  and write\_RLum2CSV  and it imports an Daybreak-file (TXT-file, DAT-file) and directly exports its content to CSV-files.  If nothing is set for the argument  path  ( write\_RLum2CSV ) the input folder will become the output folder. & 0.1.0
 &  &  & Sebastian Kreutzer, Institute of Geography, Heidelberg University (Germany)$<$br /$>$ , RLum Developer Team &  \\ 
  51 & convert\_PSL2CSV & Export PSL-file(s) to CSV-files & This function is a wrapper function around the functions  read\_PSL2R  and write\_RLum2CSV  and it imports an PSL-file (SUERC portable OSL reader file format) and directly exports its content to CSV-files. If nothing is set for the argument  path  ( write\_RLum2CSV ) the input folder will become the output folder. & 0.1.0
 &  &  & Sebastian Kreutzer, Institute of Geography, Heidelberg University (Germany)$<$br /$>$ , RLum Developer Team &  \\ 
  52 & convert\_RLum2Risoe.BINfileData & Converts RLum.Analysis-objects and RLum.Data.Curve-objects to RLum2Risoe.BINfileData-objects & The functions converts  RLum.Analysis  and  RLum.Data.Curve  objects and a  list  of those to  Risoe.BINfileData  objects. The function intends to provide a minimum of compatibility between both formats. The created  RLum.Analysis  object can be later exported to a BIN-file using the function  write\_R2BIN . & 0.1.3
 &  &  & Sebastian Kreutzer, Institute of Geography, Heidelberg University (Germany)$<$br /$>$ , RLum Developer Team &  \\ 
  53 & convert\_SG2MG & Converts Single-Grain Data to Multiple-Grain Data & Conversion of single-grain data to multiple-grain data by adding signals from grains belonging to one disc (unique pairs of position, set and run). & 0.1.0
 &  &  & Sebastian Kreutzer, Institute of Geography, Heidelberg University (Germany), Norbert Mercier, IRAMAT-CRP2A, UMR 5060, CNRS-Université Bordeaux Montaigne (France);$<$br /$>$ , RLum Developer Team &  \\ 
  54 & convert\_Wavelength2Energy & Emission Spectra Conversion from Wavelength to Energy Scales & The function provides a convenient and fast way to convert emission spectra wavelength to energy scales. The function works on  RLum.Data.Spectrum ,  data.frame  and  matrix  and a  list  of such objects. The function was written to smooth the workflow while analysing emission spectra data. This is in particular useful if you want to further treat your data and apply, e.g., a signal deconvolution. & 0.1.1
 &  &  & Sebastian Kreutzer, Institute of Geography, Heidelberg University (Germany)$<$br /$>$ , RLum Developer Team &  \\ 
  55 & convert\_XSYG2CSV & Export XSYG-file(s) to CSV-files & This function is a wrapper function around the functions  read\_XSYG2R  and write\_RLum2CSV  and it imports an XSYG-file and directly exports its content to CSV-files. If nothing is set for the argument  path  ( write\_RLum2CSV ) the input folder will become the output folder. & 0.1.0
 &  &  & Sebastian Kreutzer, Institute of Geography, Heidelberg University (Germany)$<$br /$>$ , RLum Developer Team &  \\ 
  56 & CW2pHMi & Transform a CW-OSL curve into a pHM-OSL curve via interpolation under$<$br /$>$ hyperbolic modulation conditions & This function transforms a conventionally measured continuous-wave (CW) OSL-curve to a pseudo hyperbolic modulated (pHM) curve under hyperbolic modulation conditions using the interpolation procedure described by Bos \& Wallinga (2012). & 0.2.2
 &  &  & Sebastian Kreutzer, Institute of Geography, Heidelberg University (Germany) $<$br /$>$ Based on comments and suggestions from: $<$br /$>$ Adrie J.J. Bos, Delft University of Technology, The Netherlands$<$br /$>$ , RLum Developer Team &  \\ 
  57 & CW2pLM & Transform a CW-OSL curve into a pLM-OSL curve & Transforms a conventionally measured continuous-wave (CW) curve into a pseudo linearly modulated (pLM) curve using the equations given in Bulur (2000). & 0.4.1
 &  &  & Sebastian Kreutzer, Institute of Geography, Heidelberg University (Germany)$<$br /$>$ , RLum Developer Team &  \\ 
  58 & CW2pLMi & Transform a CW-OSL curve into a pLM-OSL curve via interpolation under linear$<$br /$>$ modulation conditions & Transforms a conventionally measured continuous-wave (CW) OSL-curve into a pseudo linearly modulated (pLM) curve under linear modulation conditions using the interpolation procedure described by Bos \& Wallinga (2012). & 0.3.1
 &  &  & Sebastian Kreutzer, Institute of Geography, Heidelberg University (Germany)$<$br /$>$ $<$br /$>$ Based on comments and suggestions from: $<$br /$>$ Adrie J.J. Bos, Delft University of Technology, The Netherlands$<$br /$>$ , RLum Developer Team &  \\ 
  59 & CW2pPMi & Transform a CW-OSL curve into a pPM-OSL curve via interpolation under$<$br /$>$ parabolic modulation conditions & Transforms a conventionally measured continuous-wave (CW) OSL-curve into a pseudo parabolic modulated (pPM) curve under parabolic modulation conditions using the interpolation procedure described by Bos \& Wallinga (2012). & 0.2.1
 &  &  & Sebastian Kreutzer, Institute of Geography, Heidelberg University (Germany)$<$br /$>$ $<$br /$>$ Based on comments and suggestions from: $<$br /$>$ Adrie J.J. Bos, Delft University of Technology, The Netherlands$<$br /$>$ , RLum Developer Team &  \\ 
  60 & ExampleData.Al2O3C & Example Al2O3:C Measurement Data & Measurement data obtained from measuring Al2O3:C chips at the IRAMAT-CRP2A, Université Bordeaux Montaigne in 2017 on a Freiberg Instruments lexsyg SMART reader. The example data used in particular to allow test of the functions developed in framework of the work by Kreutzer et al., 2018. &  &  &  &  &  \\ 
  61 & ExampleData.BINfileData & Example data from a SAR OSL and SAR TL measurement for the package$<$br /$>$ Luminescence & Example data from a SAR OSL and TL measurement for package Luminescence directly extracted from a Risoe BIN-file and provided in an object of type Risoe.BINfileData &  &  &  &  &  \\ 
  62 & ExampleData.CobbleData & Example data for calc\_CobbleDoseRate() & An example data set for the function  calc\_CobbleDoseRate  containing layer specific information for the cobble to be used in the function. &  &  &  &  &  \\ 
  63 & ExampleData.CW\_OSL\_Curve & Example CW-OSL curve data for the package Luminescence & data.frame  containing CW-OSL curve data (time, counts) &  &  &  &  &  \\ 
  64 & ExampleData.DeValues & Example De data sets for the package Luminescence & Equivalent dose (De) values measured for a fine grain quartz sample from a loess section in Rottewitz (Saxony/Germany) and for a coarse grain quartz sample from a fluvial deposit in the rock shelter of Cueva Anton (Murcia/Spain). &  &  &  &  &  \\ 
  65 & ExampleData.Fading & Example data for feldspar fading measurements & Example data set for fading measurements of the IR50, IR100, IR150 and IR225 feldspar signals of sample UNIL/NB123. It further contains regular equivalent dose measurement data of the same sample, which can be used to apply a fading correction to. &  &  &  &  &  \\ 
  66 & ExampleData.FittingLM & Example data for fit\_LMCurve() in the package Luminescence & Linearly modulated (LM) measurement data from a quartz sample from Norway including background measurement. Measurements carried out in the luminescence laboratory at the University of Bayreuth. &  &  &  &  &  \\ 
  67 & ExampleData.LxTxData & Example Lx/Tx data from CW-OSL SAR measurement & LxTx data from a SAR measurement for the package Luminescence. &  &  &  &  &  \\ 
  68 & ExampleData.LxTxOSLData & Example Lx and Tx curve data from an artificial OSL measurement & Lx  and  Tx  data of continuous wave (CW-) OSL signal curves. &  &  &  &  &  \\ 
  69 & ExampleData.MortarData & Example equivalent dose data from mortar samples & Arbitrary data to test the function  calc\_EED\_Model &  &  &  &  &  \\ 
  70 & ExampleData.portableOSL & Example portable OSL curve data for the package Luminescence & A  list  of  RLum.Analysis  objects, each containing the same number of  RLum.Data.Curve  objects representing individual OSL, IRSL and dark count measurements of a sample. &  &  &  &  &  \\ 
  71 & ExampleData.RLum.Analysis & Example data as  RLum.Analysis  objects & Collection of different  RLum.Analysis  objects for protocol analysis. &  &  &  &  &  \\ 
  72 & ExampleData.RLum.Data.Image & Example data as  RLum.Data.Image  objects & Measurement of Princton Instruments camera imported with the function read\_SPE2R  to R to produce an RLum.Data.Image  object. &  &  &  &  &  \\ 
  73 & ExampleData.ScaleGammaDose & Example data for scale\_GammaDose() & An example data set for the function  scale\_GammaDose()  containing layer specific information to scale the gamma dose rate considering variations in soil radioactivity. &  &  &  &  &  \\ 
  74 & ExampleData.SurfaceExposure & Example OSL surface exposure dating data & A set of synthetic OSL surface exposure dating data to demonstrate the fit\_SurfaceExposure  functionality. See examples to reproduce the data interactively. &  &  &  &  &  \\ 
  75 & ExampleData.TR\_OSL & Example TR-OSL data & Single TR-OSL curve obtained by Schmidt et al. (under review) for quartz sample BT729 (origin: Trebgast Valley, Germany, quartz, 90-200 µm, unpublished data). &  &  &  &  &  \\ 
  76 & ExampleData.XSYG & Example data for a SAR OSL measurement and a TL spectrum using a lexsyg$<$br /$>$ reader & Example data from a SAR OSL measurement and a TL spectrum for package Luminescence imported from a Freiberg Instruments XSYG file using the function  read\_XSYG2R . &  &  &  &  &  \\ 
  77 & extdata & Collection of External Data & Description and listing of data provided in the folder  data/extdata &  &  &  &  &  \\ 
  78 & extract\_IrradiationTimes & Extract Irradiation Times from an XSYG-file & Extracts irradiation times, dose and times since last irradiation, from a Freiberg Instruments XSYG-file. These information can be further used to update an existing BINX-file. & 0.3.2
 &  &  & Sebastian Kreutzer, Institute of Geography, Heidelberg University (Germany)$<$br /$>$ , RLum Developer Team &  \\ 
  79 & extract\_ROI & Extract Pixel Values through Circular Region-of-Interests (ROI) from an Image & Light-weighted function to extract pixel values from pre-defined regions-of-interest (ROI) from RLum.Data.Image ,  array  or  matrix  objects and provide simple image processing capacity. The function is limited to circular ROIs. & 0.1.0
 &  &  & Sebastian Kreutzer, Institute of Geography, Heidelberg University (Germany)$<$br /$>$ , RLum Developer Team &  \\ 
  80 & fit\_CWCurve & Nonlinear Least Squares Fit for CW-OSL curves -beta version- & The function determines the weighted least-squares estimates of the component parameters of a CW-OSL signal for a given maximum number of components and returns various component parameters. The fitting procedure uses the  nls  function with the  port  algorithm. & 0.5.2
 &  &  & Sebastian Kreutzer, Institute of Geography, Heidelberg University (Germany)$<$br /$>$ , RLum Developer Team &  \\ 
  81 & fit\_EmissionSpectra & Luminescence Emission Spectra Deconvolution & Luminescence spectra deconvolution on  RLum.Data.Spectrum  and  matrix  objects on an  energy scale . The function is optimised for emission spectra typically obtained in the context of TL, OSL and RF measurements detected between 200 and 1000 nm. The function is not prepared to deconvolve TL curves (counts against temperature; no wavelength scale). If you are interested in such analysis, please check, e.g., the package  'tgcd' . & 0.1.1
 &  &  & Sebastian Kreutzer, Institute of Geography, Heidelberg University (Germany)$<$br /$>$ , RLum Developer Team &  \\ 
  82 & fit\_LMCurve & Nonlinear Least Squares Fit for LM-OSL curves & The function determines weighted nonlinear least-squares estimates of the component parameters of an LM-OSL curve (Bulur 1996) for a given number of components and returns various component parameters. The fitting procedure uses the function  nls  with the  port  algorithm. & 0.3.3
 &  &  & Sebastian Kreutzer, Institute of Geography, Heidelberg University (Germany)$<$br /$>$ , RLum Developer Team &  \\ 
  83 & fit\_OSLLifeTimes & Fitting and Deconvolution of OSL Lifetime Components & Fitting and Deconvolution of OSL Lifetime Components & 0.1.5
 &  &  & Sebastian Kreutzer, Geography \& Earth Sciences, Aberystwyth University,$<$br /$>$ Christoph Schmidt, University of Bayreuth (Germany)$<$br /$>$ , RLum Developer Team &  \\ 
  84 & fit\_SurfaceExposure & Nonlinear Least Squares Fit for OSL surface exposure data & This function determines the (weighted) least-squares estimates of the parameters of either equation 1 in  Sohbati et al. (2012a)  or equation 12 in Sohbati et al. (2012b)  for a given OSL surface exposure data set ( BETA ). & 0.1.0
 &  &  & Christoph Burow, University of Cologne (Germany)$<$br /$>$ , RLum Developer Team &  \\ 
  85 & fit\_ThermalQuenching & Fitting Thermal Quenching Data & Applying a nls-fitting to thermal quenching data. & 0.1.0
 &  &  & Sebastian Kreutzer, Institute of Geography, Heidelberg University (Germany)$<$br /$>$ , RLum Developer Team &  \\ 
  86 & get\_Layout & Collection of layout definitions & This helper function returns a list with layout definitions for homogeneous plotting. & 0.1
 &  &  & Michael Dietze, GFZ Potsdam (Germany)$<$br /$>$ , RLum Developer Team &  \\ 
  87 & get\_Quote & Function to return essential quotes & This function returns one of the collected essential quotes in the growing library. If called without any parameters, a random quote is returned. & 0.1.5
 &  &  & Quote credits: Michael Dietze, GFZ Potsdam (Germany), Sebastian Kreutzer, Geography \& Earth Science, Aberystwyth University (United Kingdom), Dirk Mittelstraß, TU Dresden (Germany), Jakob Wallinga (Wageningen University, Netherlands)$<$br /$>$ , RLum Developer Team &  \\ 
  88 & get\_rightAnswer & Function to get the right answer & This function returns just the right answer & 0.1.0
 &  &  & inspired by R.G.$<$br /$>$ , RLum Developer Team &  \\ 
  89 & get\_Risoe.BINfileData & General accessor function for RLum S4 class objects & Function calls object-specific get functions for RisoeBINfileData S4 class objects. & 0.1.0
 &  &  & Sebastian Kreutzer, Institute of Geography, Heidelberg University (Germany)$<$br /$>$ , RLum Developer Team &  \\ 
  90 & get\_RLum & General accessors function for RLum S4 class objects & Function calls object-specific get functions for RLum S4 class objects. & 0.3.3
 &  &  & Sebastian Kreutzer, Institute of Geography, Heidelberg University (Germany)$<$br /$>$ , RLum Developer Team &  \\ 
  91 & GitHub-API & GitHub API & R Interface to the GitHub API v3. & 0.1.0
 &  &  & Christoph Burow, University of Cologne (Germany)$<$br /$>$ , RLum Developer Team &  \\ 
  92 & install\_DevelopmentVersion & Attempts to install the development version of the 'Luminescence' package & This function is a convenient method for installing the development version of the R package 'Luminescence' directly from GitHub. &  &  &  &  &  \\ 
  93 & length\_RLum & General accessor function for RLum S4 class objects & Function calls object-specific get functions for RLum S4 class objects. & 0.1.0
 &  &  & Sebastian Kreutzer, Institute of Geography, Heidelberg University (Germany)$<$br /$>$ (France)$<$br /$>$ , RLum Developer Team &  \\ 
  94 & merge\_Risoe.BINfileData & Merge Risoe.BINfileData objects or Risoe BIN-files & Function allows merging Risoe BIN/BINX files or  Risoe.BINfileData  objects. & 0.2.7
 &  &  & Sebastian Kreutzer, Institute of Geography, Heidelberg University (Germany)$<$br /$>$ , RLum Developer Team &  \\ 
  95 & merge\_RLum.Analysis & Merge function for RLum.Analysis S4 class objects & Function allows merging of RLum.Analysis objects and adding of allowed objects to an RLum.Analysis. & 0.2.0
 &  &  & Sebastian Kreutzer, Institute of Geography, Heidelberg University (Germany)$<$br /$>$ , RLum Developer Team &  \\ 
  96 & merge\_RLum.Data.Curve & Merge function for RLum.Data.Curve S4 class objects & Function allows merging of RLum.Data.Curve objects in different ways & 0.2.1
 &  &  & Sebastian Kreutzer, Institute of Geography, Heidelberg University (Germany)$<$br /$>$ , RLum Developer Team &  \\ 
  97 & merge\_RLum & General merge function for RLum S4 class objects & Function calls object-specific merge functions for RLum S4 class objects. & 0.1.2
 &  &  & Sebastian Kreutzer, Institute of Geography, Heidelberg University (Germany)$<$br /$>$ , RLum Developer Team &  \\ 
  98 & merge\_RLum.Results & Merge function for RLum.Results S4-class objects & Function merges objects of class  RLum.Results . The slots in the objects are combined depending on the object type, e.g., for  data.frame  and  matrix  rows are appended. & 0.2.1
 &  &  & Sebastian Kreutzer, Institute of Geography, Heidelberg University (Germany)$<$br /$>$ , RLum Developer Team &  \\ 
  99 & methods\_RLum & methods\_RLum & Methods for S3-generics implemented for the package 'Luminescence'. This document summarises all implemented S3-generics. The name of the function is given before the first dot, after the dot the name of the object that is supported by this method is given, e.g.  plot.RLum.Data.Curve  can be called by  plot(object, ...) , where  object  is the  RLum.Data.Curve  object. &  &  &  &  &  \\ 
  100 & names\_RLum & S4-names function for RLum S4 class objects & Function calls object-specific names functions for RLum S4 class objects. & 0.1.0
 &  &  & Sebastian Kreutzer, Institute of Geography, Heidelberg University (Germany)$<$br /$>$ , RLum Developer Team &  \\ 
  101 & plot\_AbanicoPlot & Function to create an Abanico Plot. & A plot is produced which allows comprehensive presentation of data precision and its dispersion around a central value as well as illustration of a kernel density estimate, histogram and/or dot plot of the dose values. & 0.1.17
 &  &  & Michael Dietze, GFZ Potsdam (Germany) $<$br /$>$ Sebastian Kreutzer, Institute of Geography, Heidelberg University (Germany) $<$br /$>$ Inspired by a plot introduced by Galbraith \& Green (1990)$<$br /$>$ , RLum Developer Team &  \\ 
  102 & plot\_DetPlot & Create De(t) plot & Plots the equivalent dose (De) in dependency of the chosen signal integral (cf. Bailey et al., 2003). The function is simply passing several arguments to the function  plot  and the used analysis functions and runs it in a loop. Example:  legend.pos  for legend position,  legend  for legend text. & 0.1.3
 &  &  & Geography \& Earth Sciences, Aberystwyth University (United Kingdom)$<$br /$>$ , RLum Developer Team &  \\ 
  103 & plot\_DRCSummary & Create a Dose-Response Curve Summary Plot & While analysing OSL SAR or pIRIR-data the view on the data is limited usually to one dose-response curve (DRC) at the time for one aliquot. This function overcomes this limitation by plotting all DRC from an  RLum.Results  object created by the function  analyse\_SAR.CWOSL  in one single plot. & 0.2.2
 &  &  & Sebastian Kreutzer, Institute of Geography, Heidelberg University (Germany)  $<$br /$>$ Christoph Burow, University of Cologne$<$br /$>$ , RLum Developer Team &  \\ 
  104 & plot\_DRTResults & Visualise dose recovery test results & The function provides a standardised plot output for dose recovery test measurements. & 0.1.14
 &  &  & Sebastian Kreutzer, Institute of Geography, Heidelberg University (Germany) $<$br /$>$ Michael Dietze, GFZ Potsdam (Germany)$<$br /$>$ , RLum Developer Team &  \\ 
  105 & plot\_FilterCombinations & Plot filter combinations along with the (optional) net transmission window & The function allows to plot transmission windows for different filters. Missing data for specific wavelengths are automatically interpolated for the given filter data using the function  approx . With that a standardised output is reached and a net transmission window can be shown. & 0.3.2
 &  &  & Sebastian Kreutzer, Institute of Geography, Heidelberg University (Germany)$<$br /$>$ , RLum Developer Team &  \\ 
  106 & plot\_GrowthCurve & Fit and plot a growth curve for luminescence data (Lx/Tx against dose) & A dose response curve is produced for luminescence measurements using a regenerative or additive protocol. The function supports interpolation and extrapolation to calculate the equivalent dose. & 1.11.5
 &  &  & Sebastian Kreutzer, Institute of Geography, Heidelberg University (Germany) $<$br /$>$ Michael Dietze, GFZ Potsdam (Germany)$<$br /$>$ , RLum Developer Team &  \\ 
  107 & plot\_Histogram & Plot a histogram with separate error plot & Function plots a predefined histogram with an accompanying error plot as suggested by Rex Galbraith at the UK LED in Oxford 2010. & 0.4.5
 &  &  & Michael Dietze, GFZ Potsdam (Germany) $<$br /$>$ Sebastian Kreutzer, Institute of Geography, Heidelberg University (Germany)$<$br /$>$ , RLum Developer Team &  \\ 
  108 & plot\_KDE & Plot kernel density estimate with statistics & Plot a kernel density estimate of measurement values in combination with the actual values and associated error bars in ascending order. If enabled, the boxplot will show the usual distribution parameters (median as bold line, box delimited by the first and third quartile, whiskers defined by the extremes and outliers shown as points) and also the mean and standard deviation as pale bold line and pale polygon, respectively. & 3.6.0
 &  &  & Michael Dietze, GFZ Potsdam (Germany) $<$br /$>$ Geography \& Earth Sciences, Aberystwyth University (United Kingdom)$<$br /$>$ , RLum Developer Team &  \\ 
  109 & plot\_NRt & Visualise natural/regenerated signal ratios & This function creates a Natural/Regenerated signal vs. time (NR(t)) plot as shown in Steffen et al. 2009 &  &  &  & Christoph Burow, University of Cologne (Germany)$<$br /$>$ , RLum Developer Team &  \\ 
  110 & plot\_OSLAgeSummary & Plot Posterior OSL-Age Summary & A graphical summary of the statistical inference of an OSL age & 0.1.0
 &  &  & Anne Philippe, Université de Nantes (France),$<$br /$>$ Jean-Michel Galharret, Université de Nantes (France),$<$br /$>$ Norbert Mercier, IRAMAT-CRP2A, Université Bordeaux Montaigne (France),$<$br /$>$ Sebastian Kreutzer, Institute of Geography, Heidelberg University (Germany)$<$br /$>$ , RLum Developer Team &  \\ 
  111 & plot\_RadialPlot & Function to create a Radial Plot & A Galbraith's radial plot is produced on a logarithmic or a linear scale. & 0.5.7
 &  &  & Michael Dietze, GFZ Potsdam (Germany) $<$br /$>$ Sebastian Kreutzer, Institute of Geography, Heidelberg University (Germany) $<$br /$>$ Based on a rewritten S script of Rex Galbraith, 2010$<$br /$>$ , RLum Developer Team &  \\ 
  112 & plot\_Risoe.BINfileData & Plot single luminescence curves from a BIN file object & Plots single luminescence curves from an object returned by the read\_BIN2R  function. & 0.4.1
 &  &  & Sebastian Kreutzer, Institute of Geography, Heidelberg University (Germany) $<$br /$>$ Michael Dietze, GFZ Potsdam (Germany)$<$br /$>$ , RLum Developer Team &  \\ 
  113 & plot\_RLum.Analysis & Plot function for an RLum.Analysis S4 class object & The function provides a standardised plot output for curve data of an RLum.Analysis S4 class object & 0.3.14
 &  &  & Sebastian Kreutzer, Institute of Geography, Heidelberg University (Germany)$<$br /$>$ , RLum Developer Team &  \\ 
  114 & plot\_RLum.Data.Curve & Plot function for an RLum.Data.Curve S4 class object & The function provides a standardised plot output for curve data of an RLum.Data.Curve  S4-class object. & 0.2.6
 &  &  & Sebastian Kreutzer, Institute of Geography, Heidelberg University (Germany)$<$br /$>$ , RLum Developer Team &  \\ 
  115 & plot\_RLum.Data.Image & Plot function for an  RLum.Data.Image  S4 class object & The function provides very basic plot functionality for image data of an RLum.Data.Image  object. For more sophisticated plotting it is recommended to use other very powerful packages for image processing.  Details on the plot functions   Supported plot types:  plot.type = "plot.raster"   Uses the standard plot function of R  graphics::image . If wanted, the image is enhanced, using the argument  stretch . Possible values are  hist ,  lin , and NULL . The latter does nothing. The argument  useRaster = TRUE  is used by default, but can be set to  FALSE .  plot.type = "contour"   This uses the function  graphics::contour & 0.2.1
 &  &  & Sebastian Kreutzer, Institute of Geography, Heidelberg University (Germany)$<$br /$>$ , RLum Developer Team &  \\ 
  116 & plot\_RLum.Data.Spectrum & Plot function for an RLum.Data.Spectrum S4 class object & The function provides a standardised plot output for spectrum data of an RLum.Data.Spectrum  class object. The purpose of this function is to provide easy and straight-forward spectra plotting, not provide a full customised access to all plot parameters. If this is wanted, standard R plot functionality should be used instead.  Matrix structure    (cf.  RLum.Data.Spectrum )    rows  (x-values): wavelengths/channels ( xlim ,  xlab )   columns  (y-values): time/temperature ( ylim ,  ylab )   cells  (z-values): count values ( zlim ,  zlab )   Note: This nomenclature is valid for all plot types of this function!   Nomenclature for value limiting     xlim : Limits values along the wavelength axis   ylim : Limits values along the time/temperature axis   zlim : Limits values along the count value axis   Details on the plot functions   Spectrum is visualised as 3D or 2D plot. Both plot types are based on internal R plot functions.  plot.type = "persp"   Arguments that will be passed to  graphics::persp :    shade : default is  0.4    phi : default is  15    theta : default is  -30    expand : default is  1    axes : default is  TRUE    box : default is  TRUE ; accepts  "alternate"  for a custom plot design   ticktype : default is  detailed ,  r : default is  10    Note:  Further parameters can be adjusted via  par . For example to set the background transparent and reduce the thickness of the lines use: par(bg = NA, lwd = 0.7)  previous the function call.  plot.type = "single"   Per frame a single curve is returned. Frames are time or temperature steps.  - frames : pick the frames to be plotted (depends on the binning!). Check without this setting before plotting.  plot.type = "multiple.lines"   All frames plotted in one frame.  - frames : pick the frames to be plotted (depends on the binning!). Check without this setting before plotting.  '** plot.type = "image"  or `plot.type = "contour" **  These plot types use the R functions  graphics::image  or  graphics::contour . The advantage is that many plots can be arranged conveniently using standard R plot functionality. If  plot.type = "image"  a contour is added by default, which can be disabled using the argument  contour = FALSE  to add own contour lines of choice.  plot.type = "transect"   Depending on the selected wavelength/channel range a transect over the time/temperature (y-axis) will be plotted along the wavelength/channels (x-axis). If the range contains more than one channel, values (z-values) are summed up. To select a transect use the  xlim  argument, e.g. xlim = c(300,310)  plot along the summed up count values of channel 300 to 310.  Further arguments that will be passed (depending on the plot type)   xlab ,  ylab ,  zlab ,  xlim ,  ylim ,  box , zlim ,  main ,  mtext ,  pch ,  type  ( "single" ,  "multiple.lines" ,  "interactive" ), col ,  border ,  lwd ,  bty ,  showscale  ( "interactive" ,  "image" ) contour ,  contour.col  ( "image" ) & 0.6.8
 &  &  & Sebastian Kreutzer, Institute of Geography, Heidelberg University (Germany)$<$br /$>$ , RLum Developer Team &  \\ 
  117 & plot\_RLum & General plot function for RLum S4 class objects & Function calls object specific plot functions for RLum S4 class objects. & 0.4.4
 &  &  & Sebastian Kreutzer, Institute of Geography, Heidelberg University (Germany)$<$br /$>$ , RLum Developer Team &  \\ 
  118 & plot\_RLum.Results & Plot function for an RLum.Results S4 class object & The function provides a standardised plot output for data of an RLum.Results S4 class object & 0.2.1
 &  &  & Christoph Burow, University of Cologne (Germany)  $<$br /$>$ Sebastian Kreutzer, Institute of Geography, Heidelberg University (Germany)$<$br /$>$ , RLum Developer Team &  \\ 
  119 & plot\_ROI & Create Regions of Interest (ROI) Graphic & Create ROI graphic with data extracted from the data imported via  read\_RF2R . This function is used internally by  analyse\_IRSAR.RF  but might be of use to work with reduced data from spatially resolved measurements. The plot dimensions mimic the original image dimensions & 0.2.0
 &  &  & Sebastian Kreutzer, Department of Geography \& Earth Sciences, Aberystwyth University$<$br /$>$ (United Kingdom)$<$br /$>$ , RLum Developer Team &  \\ 
  120 & plot\_ViolinPlot & Create a violin plot & Draws a kernel density plot in combination with a boxplot in its middle. The shape of the violin is constructed using a mirrored density curve. This plot is especially designed for cases where the individual errors are zero or to small to be visualised. The idea for this plot is based on the the 'volcano plot' in the ggplot2 package by Hadley Wickham and Winston Chang. The general idea for the violin plot seems to be introduced by Hintze and Nelson (1998).  The function is passing several arguments to the function  plot , stats::density ,  graphics::boxplot :  Supported arguments are: xlim ,  main ,  xlab ,  ylab ,  col.violin ,  col.boxplot ,  mtext ,  cex ,  mtext   Valid summary keywords   'n' ,  'mean' ,  'median' ,  'sd.abs' ,  'sd.rel' ,  'se.abs' ,  'se.rel' . 'skewness' ,  'kurtosis' & 0.1.4
 &  &  & Sebastian Kreutzer, Institute of Geography, Heidelberg University (Germany)$<$br /$>$ , RLum Developer Team &  \\ 
  121 & PSL2Risoe.BINfileData & Convert portable OSL data to an Risoe.BINfileData object & Converts an  RLum.Analysis  object produced by the function  read\_PSL2R()  to an  Risoe.BINfileData  object  (BETA) . & 0.0.1
 &  &  & Christoph Burow, University of Cologne (Germany)$<$br /$>$ , RLum Developer Team &  \\ 
  122 & read\_BIN2R & Import Risø BIN/BINX-files into R & Import a *.bin or a *.binx file produced by a Risø DA15 and DA20 TL/OSL reader into R. & 0.16.5
 &  &  & Sebastian Kreutzer, Institute of Geography, Heidelberg University (Germany) $<$br /$>$ Margret C. Fuchs, HZDR Freiberg, (Germany)  $<$br /$>$ based on information provided by Torben Lapp and Karsten Bracht Nielsen (Risø DTU, Denmark)$<$br /$>$ , RLum Developer Team &  \\ 
  123 & read\_Daybreak2R & Import measurement data produced by a Daybreak TL/OSL reader into R & Import a TXT-file (ASCII file) or a DAT-file (binary file) produced by a Daybreak reader into R. The import of the DAT-files is limited to the file format described for the software TLAPLLIC v.3.2 used for a Daybreak, model 1100. & 0.3.2
 &  &  & Sebastian Kreutzer, Institute of Geography, Heidelberg University (Germany) $<$br /$>$ Antoine Zink, C2RMF, Palais du Louvre, Paris (France)$<$br /$>$ $<$br /$>$ The ASCII-file import is based on a suggestion by Willian Amidon and Andrew Louis Gorin$<$br /$>$ , RLum Developer Team &  \\ 
  124 & read\_PSL2R & Import PSL files to R & Imports PSL files produced by a SUERC portable OSL reader into R  (BETA) . & 0.0.1
 &  &  & Christoph Burow, University of Cologne (Germany)$<$br /$>$ , RLum Developer Team &  \\ 
  125 & read\_RF2R & Import RF-files to R & Import files produced by the IR-RF 'ImageJ' macro ( SR-RF.ijm ; Mittelstraß and Kreutzer, 2021) into R and create a list of  RLum.Analysis  objects & 0.1.0
 &  &  & Sebastian Kreutzer, Geography \& Earth Science, Aberystwyth University (United Kingdom)$<$br /$>$ , RLum Developer Team &  \\ 
  126 & read\_SPE2R & Import Princeton Instruments (TM) SPE-file into R & Function imports Princeton Instruments (TM) SPE-files into R environment and provides  RLum.Data.Image  objects as output. & 0.1.4
 &  &  & Sebastian Kreutzer, Institute of Geography, Heidelberg University (Germany)$<$br /$>$ , RLum Developer Team &  \\ 
  127 & read\_TIFF2R & Import TIFF Image Data into R & Simple wrapper around  tiff::readTIFF  to import TIFF images and TIFF image stacks to be further processed within the package  'Luminescence' & 0.1.1
 &  &  & Sebastian Kreutzer, Institute of Geography, Heidelberg University (Germany)$<$br /$>$ , RLum Developer Team &  \\ 
  128 & read\_XSYG2R & Import XSYG files to R & Imports XSYG-files produced by a Freiberg Instruments lexsyg reader into R. & 0.6.8
 &  &  & Sebastian Kreutzer, Institute of Geography, Heidelberg University (Germany)$<$br /$>$ , RLum Developer Team &  \\ 
  129 & replicate\_RLum & General replication function for RLum S4 class objects & Function replicates RLum S4 class objects and returns a list for this objects & 0.1.0
 &  &  & Sebastian Kreutzer, Institute of Geography, Heidelberg University (Germany)$<$br /$>$ , RLum Developer Team &  \\ 
  130 & report\_RLum & Create a HTML-report for (RLum) objects & Create a HTML-report for (RLum) objects & 0.1.4
 &  &  & Christoph Burow, University of Cologne (Germany),$<$br /$>$ Sebastian Kreutzer, Institute of Geography, Heidelberg University (Germany)  $<$br /$>$ , RLum Developer Team &  \\ 
  131 & Risoe.BINfileData-class & Class  "Risoe.BINfileData" & S4 class object for luminescence data in R. The object is produced as output of the function  read\_BIN2R . & 0.4.0
 &  &  & Sebastian Kreutzer, Institute of Geography, Heidelberg University (Germany) $<$br /$>$ based on information provided by Torben Lapp and Karsten Bracht Nielsen (Risø DTU, Denmark)$<$br /$>$ , RLum Developer Team &  \\ 
  132 & Risoe.BINfileData2RLum.Analysis & Convert Risoe.BINfileData object to an RLum.Analysis object & Converts values from one specific position of a Risoe.BINfileData S4-class object to an RLum.Analysis object. & 0.4.2
 &  &  & Sebastian Kreutzer, Institute of Geography, Heidelberg University (Germany)$<$br /$>$ , RLum Developer Team &  \\ 
  133 & RLum-class & Class  "RLum" & Abstract class for data in the package Luminescence Subclasses are: &  &  &  & Sebastian Kreutzer, Institute of Geography, Heidelberg University (Germany)$<$br /$>$ , RLum Developer Team &  \\ 
  134 & RLum.Analysis-class & Class  "RLum.Analysis" & Object class to represent analysis data for protocol analysis, i.e. all curves, spectra etc. from one measurements. Objects from this class are produced, by e.g.  read\_XSYG2R ,  read\_Daybreak2R &  &  &  & Sebastian Kreutzer, Institute of Geography, Heidelberg University (Germany)$<$br /$>$ , RLum Developer Team &  \\ 
  135 & RLum.Data-class & Class  "RLum.Data" & Generalized virtual data class for luminescence data. &  &  &  & Sebastian Kreutzer, Institute of Geography, Heidelberg University (Germany)$<$br /$>$ , RLum Developer Team &  \\ 
  136 & RLum.Data.Curve-class & Class  "RLum.Data.Curve" & Class for representing luminescence curve data. &  &  &  & Sebastian Kreutzer, IRAMAT-CRP2A, UMR 5060, CNRS - Université Bordeaux Montaigne (France)$<$br /$>$ , RLum Developer Team &  \\ 
  137 & RLum.Data.Image-class & Class  "RLum.Data.Image" & Class for representing luminescence image data (TL/OSL/RF). Such data are for example produced by the function  read\_SPE2R &  &  &  & Sebastian Kreutzer, Institute of Geography, Heidelberg University (Germany)$<$br /$>$ , RLum Developer Team &  \\ 
  138 & RLum.Data.Spectrum-class & Class  "RLum.Data.Spectrum" & Class for representing luminescence spectra data (TL/OSL/RF). &  &  &  & Sebastian Kreutzer, Institute of Geography, Heidelberg University (Germany)$<$br /$>$ , RLum Developer Team &  \\ 
  139 & RLum.Results-class & Class  "RLum.Results" & Object class contains results data from functions (e.g.,  analyse\_SAR.CWOSL ). &  &  &  & Sebastian Kreutzer, Institute of Geography, Heidelberg University (Germany)$<$br /$>$ , RLum Developer Team &  \\ 
  140 & scale\_GammaDose & Calculate the gamma dose deposited within a sample taking layer-to-layer$<$br /$>$ variations in radioactivity into account (according to Aitken, 1985) & This function calculates the gamma dose deposited in a luminescence sample taking into account layer-to-layer variations in sediment radioactivity . The function scales user inputs of uranium, thorium and potassium based on input parameters for sediment density, water content and given layer thicknesses and distances to the sample. & 0.1.2
 &  &  & Svenja Riedesel, Aberystwyth University (United Kingdom)  $<$br /$>$ Martin Autzen, DTU NUTECH Center for Nuclear Technologies (Denmark)  $<$br /$>$ Christoph Burow, University of Cologne (Germany)  $<$br /$>$ Based on an excel spreadsheet and accompanying macro written by Ian Bailiff.$<$br /$>$ , RLum Developer Team &  \\ 
  141 & Second2Gray & Converting equivalent dose values from seconds (s) to Gray (Gy) & Conversion of absorbed radiation dose in seconds (s) to the SI unit Gray (Gy) including error propagation. Normally used for equivalent dose data. & 0.6.0
 &  &  & Sebastian Kreutzer, Institute of Geography, Heidelberg University (Germany) $<$br /$>$ Michael Dietze, GFZ Potsdam (Germany) $<$br /$>$ Margret C. Fuchs, HZDR, Helmholtz-Institute Freiberg for Resource Technology (Germany)$<$br /$>$ , RLum Developer Team &  \\ 
  142 & set\_Risoe.BINfileData & General accessor function for RLum S4 class objects & Function calls object-specific get functions for RisoeBINfileData S4 class objects. & 0.1
 &  &  & Sebastian Kreutzer, Institute of Geography, Heidelberg University (Germany)$<$br /$>$ , RLum Developer Team &  \\ 
  143 & set\_RLum & General set function for RLum S4 class objects & Function calls object-specific set functions for RLum S4 class objects. & 0.3.0
 &  &  & Sebastian Kreutzer, Institute of Geography, Heidelberg University (Germany)$<$br /$>$ , RLum Developer Team &  \\ 
  144 & smooth\_RLum & Smoothing of data & Function calls the object-specific smooth functions for provided RLum S4-class objects. & 0.1.0
 &  &  & Sebastian Kreutzer, Institute of Geography, Heidelberg University (Germany)$<$br /$>$ , RLum Developer Team &  \\ 
  145 & sTeve & sTeve - sophisticated tool for efficient data validation and evaluation & This function provides a sophisticated routine for comprehensive luminescence dating data analysis. &  &  &  & R Luminescence Team, 2012-2046$<$br /$>$ , RLum Developer Team &  \\ 
  146 & structure\_RLum & General structure function for RLum S4 class objects & Function calls object-specific get functions for RLum S4 class objects. & 0.2.0
 &  &  & Sebastian Kreutzer, Institute of Geography, Heidelberg University (Germany)$<$br /$>$ , RLum Developer Team &  \\ 
  147 & template\_DRAC & Create a DRAC input data template (v1.2) & This function returns a DRAC input template (v1.2) to be used in conjunction with the  use\_DRAC  function &  &  &  & Christoph Burow, University of Cologne (Germany), Sebastian Kreutzer, Institute of Geography, Heidelberg University (Germany)$<$br /$>$ , RLum Developer Team &  \\ 
  148 & tune\_Data & Tune data for experimental purpose & The error can be reduced and sample size increased for specific purpose. & 0.5.0
 &  &  & Michael Dietze, GFZ Potsdam (Germany)$<$br /$>$ , RLum Developer Team &  \\ 
  149 & use\_DRAC & Use DRAC to calculate dose rate data & The function provides an interface from R to DRAC. An R-object or a pre-formatted XLS/XLSX file is passed to the DRAC website and the results are re-imported into R. & 0.14
 &  &  & Sebastian Kreutzer, Institute of Geography, Heidelberg University (Germany) $<$br /$>$ Michael Dietze, GFZ Potsdam (Germany) $<$br /$>$ Christoph Burow, University of Cologne (Germany)$<$br /$>$ , RLum Developer Team &  \\ 
  150 & verify\_SingleGrainData & Verify single grain data sets and check for invalid grains, i.e.$<$br /$>$ zero-light level grains & This function tries to identify automatically zero-light level curves (grains) from single grain data measurements. & 0.2.3
 &  &  & Sebastian Kreutzer, Institute of Geography, Heidelberg University (Germany)$<$br /$>$ , RLum Developer Team &  \\ 
  151 & write\_R2BIN & Export Risoe.BINfileData into Risø BIN/BINX-file & Exports a Risoe.BINfileData object in a *.bin or *.binx file that can be opened by the Analyst software or other Risø software. & 0.5.1
 &  &  & Sebastian Kreutzer, Institute of Geography, Heidelberg University (Germany)$<$br /$>$ , RLum Developer Team &  \\ 
  152 & write\_R2TIFF & Export RLum.Data.Image and RLum.Data.Spectrum objects to TIFF Images & Simple wrapper around  tiff::writeTIFF  to export suitable RLum-class objects to TIFF images. Per default 16-bit TIFF files are exported. & 0.1.0
 &  &  & Sebastian Kreutzer, Institute of Geography, Heidelberg University (Germany)$<$br /$>$ , RLum Developer Team &  \\ 
  153 & write\_RLum2CSV & Export RLum-objects to CSV & This function exports  RLum -objects to CSV-files using the R function utils::write.table . All  RLum -objects are supported, but the export is lossy, i.e. the pure numerical values are exported only. Information that cannot be coerced to a  data.frame  or a  matrix  are discarded as well as metadata. & 0.2.0
 &  &  & Sebastian Kreutzer, Geography \& Earth Science, Aberystwyth University (United Kingdom)$<$br /$>$ , RLum Developer Team &  \\ 
   \hline
\end{tabular}
\end{table}

