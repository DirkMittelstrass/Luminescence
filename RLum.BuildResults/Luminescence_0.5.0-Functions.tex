% latex table generated in R 3.2.2 by xtable 1.8-0 package
% Wed Nov 11 23:54:56 2015
\begin{table}[ht]
\centering
\begin{tabular}{rlllllll}
  \hline
 & Name & Title & Description & Version & m.Date & m.Time & Author \\ 
  \hline
1 & analyse\_IRSAR.RF & Analyse IRSAR RF measurements & $<$br /$>$ Function to analyse IRSAR RF measurements on K-feldspar samples, performed$<$br /$>$ using the protocol according to Erfurt et al. (2003) and beyond.$<$br /$>$ & 0.5.1 & 2015-10-31 & 14:56:58
 & $<$br /$>$ Sebastian Kreutzer, IRAMAT-CRP2A, Universite Bordeaux Montaigne (France)$<$br /$>$  R Luminescence Package Team \\ 
  2 & analyse\_pIRIRSequence & Analyse post-IR IRSL sequences & $<$br /$>$ The function performs an analysis of post-IR IRSL sequences including curve$<$br /$>$ fitting on  RLum.Analysis  objects.$<$br /$>$ & 0.1.5 & 2015-11-11 & 18:18:13
 & $<$br /$>$ Sebastian Kreutzer, IRAMAT-CRP2A, Universite Bordeaux Montaigne$<$br /$>$ (France)$<$br /$>$  R Luminescence Package Team \\ 
  3 & analyse\_SAR.CWOSL & Analyse SAR CW-OSL measurements & $<$br /$>$ The function performs a SAR CW-OSL analysis on an$<$br /$>$ RLum.Analysis  object including growth curve fitting.$<$br /$>$ & 0.6.4 & 2015-11-11 & 18:18:13
 & $<$br /$>$ Sebastian Kreutzer, IRAMAT-CRP2A, Universite Bordeaux Montaigne$<$br /$>$ (France)$<$br /$>$  R Luminescence Package Team \\ 
  4 & Analyse\_SAR.OSLdata & Analyse SAR CW-OSL measurements. & $<$br /$>$ The function analyses SAR CW-OSL curve data and provides a summary of the$<$br /$>$ measured data for every position. The output of the function is optimised$<$br /$>$ for SAR OSL measurements on quartz.$<$br /$>$ & 0.2.16 & 2015-10-31 & 22:27:15
 & $<$br /$>$ Sebastian Kreutzer, IRAMAT-CRP2A, Universite Bordeaux Montaigne$<$br /$>$ (France), Margret C. Fuchs, HZDR, Freiberg (Germany)$<$br /$>$  R Luminescence Package Team \\ 
  5 & analyse\_SAR.TL & Analyse SAR TL measurements & $<$br /$>$ The function performs a SAR TL analysis on a$<$br /$>$ RLum.Analysis  object including growth curve fitting.$<$br /$>$ & 0.1.4 & 2015-10-17 & 14:30:28
 & $<$br /$>$ Sebastian Kreutzer, IRAMAT-CRP2A, Universite Bordeaux Montaigne (France)$<$br /$>$  R Luminescence Package Team \\ 
  6 & apply\_CosmicRayRemoval & Function to remove cosmic rays from an RLum.Data.Spectrum S4 class object & $<$br /$>$ The function provides several methods for cosmic ray removal and spectrum$<$br /$>$ smoothing for an RLum.Data.Spectrum S4 class object$<$br /$>$ & 0.1.3 & 2015-09-22 & 09:41:30
 & $<$br /$>$ Sebastian Kreutzer, IRAMAT-CRP2A, Universite Bordeaux Montaigne$<$br /$>$ (France)$<$br /$>$  R Luminescence Package Team \\ 
  7 & apply\_EfficiencyCorrection & Function to apply spectral efficiency correction to RLum.Data.Spectrum S4$<$br /$>$ class objects & $<$br /$>$ The function allows spectral efficiency corrections for RLum.Data.Spectrum$<$br /$>$ S4 class objects$<$br /$>$ & 0.1 & 2015-09-22 & 09:41:30
 & $<$br /$>$ Sebastian Kreutzer, IRAMAT-CRP2A, Universite Bordeaux Montaigne$<$br /$>$ (France),  Johannes Friedrich, University of Bayreuth (Germany)$<$br /$>$  R Luminescence Package Team \\ 
  8 & BaseDataSet.CosmicDoseRate & Base data set for cosmic dose rate calculation & $<$br /$>$ Collection of data from various sources needed for cosmic dose rate$<$br /$>$ calculation$<$br /$>$ &  &  &  &  \\ 
  9 & calc\_AliquotSize & Estimate the amount of grains on an aliquot & $<$br /$>$ Estimate the number of grains on an aliquot. Alternatively, the packing$<$br /$>$ density of an aliquot is computed.$<$br /$>$ & 0.31 & 2015-09-22 & 09:41:30
 & $<$br /$>$ Christoph Burow, University of Cologne (Germany)$<$br /$>$  R Luminescence Package Team \\ 
  10 & calc\_CentralDose & Apply the central age model (CAM) after Galbraith et al. (1999) to a given$<$br /$>$ De distribution & $<$br /$>$ This function calculates the central dose and dispersion of the De$<$br /$>$ distribution, their standard errors and the profile log likelihood function$<$br /$>$ for sigma.$<$br /$>$ & 1.3.1 & 2015-10-06 & 23:28:25
 & $<$br /$>$ Christoph Burow, University of Cologne (Germany)   Based on a$<$br /$>$ rewritten S script of Rex Galbraith, 2010  $<$br /$>$  R Luminescence Package Team \\ 
  11 & calc\_CommonDose & Apply the (un-)logged common age model after Galbraith et al. (1999) to a$<$br /$>$ given De distribution & $<$br /$>$ Function to calculate the common dose of a De distribution.$<$br /$>$ & 0.1 & 2015-09-22 & 09:41:30
 & $<$br /$>$ Christoph Burow, University of Cologne (Germany)$<$br /$>$  R Luminescence Package Team \\ 
  12 & calc\_CosmicDoseRate & Calculate the cosmic dose rate & $<$br /$>$ This function calculates the cosmic dose rate taking into account the soft-$<$br /$>$ and hard-component of the cosmic ray flux and allows corrections for$<$br /$>$ geomagnetic latitude, altitude above sea-level and geomagnetic field$<$br /$>$ changes.$<$br /$>$ & 0.5.2 & 2015-09-22 & 09:41:30
 & $<$br /$>$ Christoph Burow, University of Cologne (Germany)$<$br /$>$  R Luminescence Package Team \\ 
  13 & calc\_FadingCorr & Apply a fading correction according to Huntley \& Lamothe (2001) for a given$<$br /$>$ g-value. & $<$br /$>$ This function runs the iterations that are needed to calculate the corrected$<$br /$>$ age including the error for a given g-value according to Huntley \& Lamothe$<$br /$>$ (2001).$<$br /$>$ & 0.3.0 & 2015-09-22 & 09:41:30
 & $<$br /$>$ Sebastian Kreutzer, IRAMAT-CRP2A, Universite Bordeaux Montaigne$<$br /$>$  R Luminescence Package Team \\ 
  14 & calc\_FiniteMixture & Apply the finite mixture model (FMM) after Galbraith (2005) to a given De$<$br /$>$ distribution & $<$br /$>$ This function fits a k-component mixture to a De distribution with differing$<$br /$>$ known standard errors. Parameters (doses and mixing proportions) are$<$br /$>$ estimated by maximum likelihood assuming that the log dose estimates are$<$br /$>$ from a mixture of normal distributions.$<$br /$>$ & 0.4 & 2015-09-22 & 09:41:30
 & $<$br /$>$ Christoph Burow, University of Cologne (Germany)   Based on a$<$br /$>$ rewritten S script of Rex Galbraith, 2006.  $<$br /$>$  R Luminescence Package Team \\ 
  15 & calc\_FuchsLang2001 & Apply the model after Fuchs \& Lang (2001) to a given De distribution. & $<$br /$>$ This function applies the method according to Fuchs \& Lang (2001) for$<$br /$>$ heterogeneously bleached samples with a given coefficient of variation$<$br /$>$ threshold.$<$br /$>$ & 0.4.1 & 2015-09-22 & 09:41:30
 & $<$br /$>$ Sebastian Kreutzer, IRAMAT-CRP2A, Universite Bordeaux Montaigne$<$br /$>$ (France) Christoph Burow, University of Cologne (Germany)$<$br /$>$  R Luminescence Package Team \\ 
  16 & calc\_gSGC & Calculate De value based on the gSGC by Li et al., 2015 & $<$br /$>$ Function returns De value and De value error using the global standardised growth$<$br /$>$ curve (gSGC) assumption proposed by Li et al., 2015 for OSL dating of sedimentary quartz$<$br /$>$ & 0.1.0 & 2015-09-22 & 09:41:30
 & $<$br /$>$ Sebastian Kreutzer, IRAMAT-CRP2A, Universite Bordeaux Montagine (France) $<$br /$>$  R Luminescence Package Team \\ 
  17 & calc\_HomogeneityTest & Apply a simple homogeneity test after Galbraith (2003) & $<$br /$>$ A simple homogeneity test for De estimates$<$br /$>$ & 0.2 & 2015-09-22 & 09:41:30
 & $<$br /$>$ Christoph Burow, University of Cologne (Germany)$<$br /$>$  R Luminescence Package Team \\ 
  18 & calc\_IEU & Apply the internal-external-uncertainty (IEU) model after Thomsen et al.$<$br /$>$ (2007) to a given De distribution & $<$br /$>$ Function to calculate the IEU De for a De data set.$<$br /$>$ & 0.1.0 & 2015-09-22 & 09:41:30
 & $<$br /$>$ Rachel Smedley, Geography \& Earth Sciences, Aberystwyth University$<$br /$>$ (United Kingdom)   Based on an excel spreadsheet and accompanying macro$<$br /$>$ written by Kristina Thomsen.$<$br /$>$  R Luminescence Package Team \\ 
  19 & calc\_MaxDose & Apply the maximum age model to a given De distribution & $<$br /$>$ Function to fit the maximum age model to De data. This is a wrapper function$<$br /$>$ that calls calc\_MinDose() and applies a similiar approach as described in$<$br /$>$ Olley et al. (2006).$<$br /$>$ & 0.3 & 2015-09-22 & 09:41:30
 & $<$br /$>$ Christoph Burow, University of Cologne (Germany)   Based on a$<$br /$>$ rewritten S script of Rex Galbraith, 2010  $<$br /$>$  R Luminescence Package Team \\ 
  20 & calc\_MinDose & Apply the (un-)logged minimum age model (MAM) after Galbraith et al. (1999)$<$br /$>$ to a given De distribution & $<$br /$>$ Function to fit the (un-)logged three or four parameter minimum dose model$<$br /$>$ (MAM-3/4) to De data.$<$br /$>$ & 0.4.1 & 2015-09-22 & 09:41:30
 & $<$br /$>$ Christoph Burow, University of Cologne (Germany)   Based on a$<$br /$>$ rewritten S script of Rex Galbraith, 2010   The bootstrap approach is$<$br /$>$ based on a rewritten MATLAB script of Alastair Cunningham.   Alastair$<$br /$>$ Cunningham is thanked for his help in implementing and cross-checking the$<$br /$>$ code.$<$br /$>$  R Luminescence Package Team \\ 
  21 & calc\_OSLLxTxRatio & Calculate Lx/Tx ratio for CW-OSL curves & $<$br /$>$ Calculate Lx/Tx ratios from a given set of CW-OSL curves assuming late light background subtraction.$<$br /$>$ & 0.5.2 & 2015-09-24 & 16:32:32
 & $<$br /$>$ Sebastian Kreutzer, IRAMAT-CRP2A, Universite Bordeaux Montaigne$<$br /$>$ (France)$<$br /$>$  R Luminescence Package Team \\ 
  22 & calc\_SourceDoseRate & Calculation of the source dose rate via the date of measurement & $<$br /$>$ Calculating the dose rate of the irradiation source via the date of$<$br /$>$ measurement based on: source calibration date, source dose rate, dose rate$<$br /$>$ error. The function returns a data.frame that provides the input argument$<$br /$>$ dose\_rate for the function  Second2Gray .$<$br /$>$ & 0.3.0 & 2015-09-22 & 09:41:30
 & $<$br /$>$ Margret C. Fuchs, HZDR, Helmholtz-Institute Freiberg for Resource Technology (Germany),$<$br /$>$  Sebastian Kreutzer, IRAMAT-CRP2A, Universite Bordeaux Montaigne (France)$<$br /$>$  R Luminescence Package Team \\ 
  23 & calc\_Statistics & Function to calculate statistic measures & $<$br /$>$ This function calculates a number of descriptive statistics for De-data,$<$br /$>$ most fundamentally using error-weighted approaches.$<$br /$>$ & 0.1.3 & 2015-09-22 & 09:41:30
 & $<$br /$>$ Michael Dietze, GFZ Potsdam (Germany)$<$br /$>$  R Luminescence Package Team \\ 
  24 & calc\_TLLxTxRatio & Calculate the Lx/Tx ratio for a given set of TL curves [beta version] & $<$br /$>$ Calculate Lx/Tx ratio for a given set of TL curves.$<$br /$>$ & 0.3.0 & 2015-10-16 & 18:19:05
 & $<$br /$>$ Sebastian Kreutzer, IRAMAT-CRP2A, Universite Bordeaux Montaigne$<$br /$>$ (France), Christoph Schmidt, University of Bayreuth (Germany)$<$br /$>$  R Luminescence Package Team \\ 
  25 & CW2pHMi & Transform a CW-OSL curve into a pHM-OSL curve via interpolation under$<$br /$>$ hyperbolic modulation conditions & $<$br /$>$ This function transforms a conventionally measured continuous-wave (CW)$<$br /$>$ OSL-curve to a pseudo hyperbolic modulated (pHM) curve under hyperbolic$<$br /$>$ modulation conditions using the interpolation procedure described by Bos \&$<$br /$>$ Wallinga (2012).$<$br /$>$ & 0.2.2 & 2015-09-22 & 09:41:30
 & $<$br /$>$ Sebastian Kreutzer, IRAMAT-CRP2A, Universite Bordeaux Montaigne$<$br /$>$ (France)   Based on comments and suggestions from:  Adrie J.J. Bos,$<$br /$>$ Delft University of Technology, The Netherlands $<$br /$>$  R Luminescence Package Team \\ 
  26 & CW2pLM & Transform a CW-OSL curve into a pLM-OSL curve & $<$br /$>$ Transforms a conventionally measured continuous-wave (CW) curve into a$<$br /$>$ pseudo linearly modulated (pLM) curve using the equations given in Bulur$<$br /$>$ (2000).$<$br /$>$ & 0.4.1 & 2015-09-22 & 09:41:30
 & $<$br /$>$ Sebastian Kreutzer, IRAMAT-CRP2A, Universite Bordeaux Montaigne$<$br /$>$ (France)$<$br /$>$  R Luminescence Package Team \\ 
  27 & CW2pLMi & Transform a CW-OSL curve into a pLM-OSL curve via interpolation under linear$<$br /$>$ modulation conditions & $<$br /$>$ Transforms a conventionally measured continuous-wave (CW) OSL-curve into a$<$br /$>$ pseudo linearly modulated (pLM) curve under linear modulation conditions$<$br /$>$ using the interpolation procedure described by Bos \& Wallinga (2012).$<$br /$>$ & 0.3.1 & 2015-09-22 & 09:41:30
 & $<$br /$>$ Sebastian Kreutzer, IRAMAT-CRP2A, Universite Bordeaux$<$br /$>$ Montaigne  Based on comments and suggestions from:  Adrie J.J. Bos,$<$br /$>$ Delft University of Technology, The Netherlands $<$br /$>$  R Luminescence Package Team \\ 
  28 & CW2pPMi & Transform a CW-OSL curve into a pPM-OSL curve via interpolation under$<$br /$>$ parabolic modulation conditions & $<$br /$>$ Transforms a conventionally measured continuous-wave (CW) OSL-curve into a$<$br /$>$ pseudo parabolic modulated (pPM) curve under parabolic modulation conditions$<$br /$>$ using the interpolation procedure described by Bos \& Wallinga (2012).$<$br /$>$ & 0.2.1 & 2015-09-22 & 09:41:30
 & $<$br /$>$ Sebastian Kreutzer, IRAMAT-CRP2A, Universite Bordeaux Montaigne$<$br /$>$ (France)  Based on comments and suggestions from:  Adrie J.J. Bos,$<$br /$>$ Delft University of Technology, The Netherlands $<$br /$>$  R Luminescence Package Team \\ 
  29 & ExampleData.BINfileData & Example data from a SAR OSL and SAR TL measurement for the package$<$br /$>$ Luminescence & $<$br /$>$ Example data from a SAR OSL and TL measurement for package Luminescence$<$br /$>$ directly extracted from a Risoe BIN-file and provided in an object of type$<$br /$>$ Risoe.BINfileData-class $<$br /$>$ &  &  &  &  \\ 
  30 & ExampleData.CW\_OSL\_Curve & Example CW-OSL curve data for the package Luminescence & $<$br /$>$ data.frame  containing CW-OSL curve data (time, counts)$<$br /$>$ &  &  &  &  \\ 
  31 & ExampleData.DeValues & Example De data sets for the package Luminescence & $<$br /$>$ Equivalent dose (De) values measured for a fine grain quartz sample from a$<$br /$>$ loess section in Rottewitz (Saxony/Germany) and for a coarse grain quartz$<$br /$>$ sample from a fluvial deposit in the rock shelter of Cueva Anton$<$br /$>$ (Murcia/Spain).$<$br /$>$ &  &  &  &  \\ 
  32 & ExampleData.FittingLM & Example data for fit\_LMCurve() in the package Luminescence & $<$br /$>$ Lineraly modulated (LM) measurement data from a quartz sample from Norway$<$br /$>$ including background measurement. Measurements carried out in the$<$br /$>$ luminescence laboratory at the University of Bayreuth.$<$br /$>$ &  &  &  &  \\ 
  33 & ExampleData.LxTxData & Example Lx/Tx data from CW-OSL SAR measurement & $<$br /$>$ LxTx data from a SAR measurement for the package Luminescence.$<$br /$>$ &  &  &  &  \\ 
  34 & ExampleData.LxTxOSLData & Example Lx and Tx curve data from an artificial OSL measurement & $<$br /$>$ Lx and Tx data of continous wave (CW-) OSL signal curves.$<$br /$>$ &  &  &  &  \\ 
  35 & ExampleData.RLum.Analysis & Example data as  RLum.Analysis  objects & $<$br /$>$ Collection of different  RLum.Analysis  objects for$<$br /$>$ protocol analysis.$<$br /$>$ &  &  &  &  \\ 
  36 & ExampleData.RLum.Data.Image & Example data as  RLum.Data.Image  objects & $<$br /$>$ Measurement of Princton Instruments camera imported with the function$<$br /$>$ read\_SPE2R  to R to produce an$<$br /$>$ RLum.Data.Image  object.$<$br /$>$ &  &  &  &  \\ 
  37 & ExampleData.XSYG & Example data for a SAR OSL measurement and a TL spectrum using a lexsyg$<$br /$>$ reader & $<$br /$>$ Example data from a SAR OSL measurement and a TL spectrum for package$<$br /$>$ Luminescence imported from a Freiberg Instruments XSYG file using the$<$br /$>$ function  read\_XSYG2R .$<$br /$>$ &  &  &  &  \\ 
  38 & extract\_IrradiationTimes & Extract irradiation times from an XSYG file & $<$br /$>$ Extracts irradiation times, dose and times since last irradiation, from a$<$br /$>$ Freiberg Instruments XSYG-file. These information can be further used to$<$br /$>$ update an existing BINX-file$<$br /$>$ & 0.2.1 & 2015-10-15 & 20:56:50
 & $<$br /$>$ Sebastian Kreutzer, IRAMAT-CRP2A, Universite Bordeaux Montaigne$<$br /$>$ (France)$<$br /$>$  R Luminescence Package Team \\ 
  39 & fit\_CWCurve & Nonlinear Least Squares Fit for CW-OSL curves [beta version] & $<$br /$>$ The function determines the weighted least-squares estimates of the$<$br /$>$ component parameters of a CW-OSL signal for a given maximum number of$<$br /$>$ components and returns various component parameters. The fitting procedure$<$br /$>$ uses the  nls  function with the  port  algorithm.$<$br /$>$ & 0.5.1 & 2015-09-22 & 09:41:30
 & $<$br /$>$ Sebastian Kreutzer, IRAMAT-CRP2A, Universite Bordeaux Montaigne$<$br /$>$ (France)$<$br /$>$  R Luminescence Package Team \\ 
  40 & fit\_LMCurve & Nonlinear Least Squares Fit for LM-OSL curves & $<$br /$>$ The function determines weighted nonlinear least-squares estimates of the$<$br /$>$ component parameters of an LM-OSL curve (Bulur 1996) for a given number of$<$br /$>$ components and returns various component parameters. The fitting procedure$<$br /$>$ uses the function  nls  with the  port  algorithm.$<$br /$>$ & 0.3.0 & 2015-09-22 & 09:41:30
 & $<$br /$>$ Sebastian Kreutzer, IRAMAT-CRP2A, Universite Bordeaux Montaigne$<$br /$>$ (France)$<$br /$>$  R Luminescence Package Team \\ 
  41 & get\_Layout & Collection of layout definitions & $<$br /$>$ This helper function returns a list with layout definitions for homogeneous$<$br /$>$ plotting.$<$br /$>$ & 0.1 & 2015-09-22 & 09:41:30
 & $<$br /$>$ Michael Dietze, GFZ Potsdam (Germany)$<$br /$>$  R Luminescence Package Team \\ 
  42 & get\_Quote & Function to return essential quotes & $<$br /$>$ This function returns one of the collected essential quotes in the$<$br /$>$ growing library. If called without any parameters, a random quote is$<$br /$>$ returned.$<$br /$>$ & 0.1.1 & 2015-11-11 & 18:18:13
 & $<$br /$>$ Michael Dietze, GFZ Potsdam (Germany)$<$br /$>$  R Luminescence Package Team \\ 
  43 & get\_rightAnswer & Function to get the right answer & $<$br /$>$ This function returns just the right answer$<$br /$>$ & 0.1.0 & 2015-09-24 & 16:32:32
 & $<$br /$>$ inspired by R.G.$<$br /$>$  R Luminescence Package Team \\ 
  44 & get\_Risoe.BINfileData & General accessor function for RLum S4 class objects & $<$br /$>$ Function calls object-specific get functions for RisoeBINfileData S4 class objects.$<$br /$>$ & 0.1.0 & 2015-10-02 & 19:04:55
 & $<$br /$>$ Sebastian Kreutzer, IRAMAT-CRP2A, Universite Bordeaux Montaigne$<$br /$>$ (France)$<$br /$>$  R Luminescence Package Team \\ 
  45 & get\_RLum & General accessor function for RLum S4 class objects & $<$br /$>$ Function calls object-specific get functions for RLum S4 class objects.$<$br /$>$ & 0.2.0 & 2015-10-28 & 10:50:06
 & $<$br /$>$ Sebastian Kreutzer, IRAMAT-CRP2A, Universite Bordeaux Montaigne$<$br /$>$ (France)$<$br /$>$  R Luminescence Package Team \\ 
  46 & length\_RLum & General accessor function for RLum S4 class objects & $<$br /$>$ Function calls object-specific get functions for RLum S4 class objects.$<$br /$>$ & 0.1 & 2015-09-22 & 09:41:30
 & $<$br /$>$ Sebastian Kreutzer, IRAMAT-CRP2A, Universite Bordeaux Montaigne$<$br /$>$ (France)$<$br /$>$  R Luminescence Package Team \\ 
  47 & merge\_Risoe.BINfileData & Merge Risoe.BINfileData objects or Risoe BIN-files & $<$br /$>$ Function allows merging Risoe BIN/BINX files or Risoe.BINfileData objects.$<$br /$>$ & 0.2.4 & 2015-10-03 & 15:51:11
 & $<$br /$>$ Sebastian Kreutzer, IRAMAT-CRP2A, Universite Bordeaux Montaigne$<$br /$>$ (France)$<$br /$>$  R Luminescence Package Team \\ 
  48 & merge\_RLum.Analysis & Merge function for RLum.Analysis S4 class objects & $<$br /$>$ Function allows merging of RLum.Analysis objects and adding of allowed$<$br /$>$ objects to an RLum.Analysis.$<$br /$>$ & 0.1 & 2015-09-22 & 09:41:30
 & $<$br /$>$ Sebastian Kreutzer, IRAMAT-CRP2A, Universite Bordeaux Montaigne$<$br /$>$ (France)$<$br /$>$  R Luminescence Package Team \\ 
  49 & merge\_RLum.Data.Curve & Merge function for RLum.Data.Curve S4 class objects & $<$br /$>$ Function allows merging of RLum.Data.Curve objects in different ways$<$br /$>$ & 0.1.1 & 2015-10-06 & 23:28:52
 & $<$br /$>$ Sebastian Kreutzer, IRAMAT-CRP2A, Universite Bordeaux Montaigne$<$br /$>$ (France)$<$br /$>$  R Luminescence Package Team \\ 
  50 & merge\_RLum & General merge function for RLum S4 class objects & $<$br /$>$ Function calls object-specific merge functions for RLum S4 class objects.$<$br /$>$ & 0.1.2 & 2015-09-22 & 09:41:30
 & $<$br /$>$ Sebastian Kreutzer, IRAMAT-CRP2A, Universite Bordeaux Montaigne$<$br /$>$ (France)$<$br /$>$  R Luminescence Package Team \\ 
  51 & merge\_RLum.Results & General accessor function for RLum S4 class objects & $<$br /$>$ Function calls object-specific get functions for RLum S4 class objects.$<$br /$>$ & 0.1
 &  &  & $<$br /$>$ Sebastian Kreutzer, IRAMAT-CRP2A, Universite Bordeaux Montaigne$<$br /$>$ (France)$<$br /$>$  R Luminescence Package Team \\ 
  52 & names\_RLum & S4-names function for RLum S4 class objects & $<$br /$>$ Function calls object-specific names functions for RLum S4 class objects.$<$br /$>$ & 0.1.0 & 2015-10-31 & 22:27:15
 & $<$br /$>$ Sebastian Kreutzer, IRAMAT-CRP2A, Universite Bordeaux Montaigne$<$br /$>$ (France)$<$br /$>$  R Luminescence Package Team \\ 
  53 & plot\_AbanicoPlot & Function to create an Abanico Plot. & $<$br /$>$ A plot is produced which allows comprehensive presentation of data precision$<$br /$>$ and its dispersion around a central value as well as illustration of a$<$br /$>$ kernel density estimate, histogram and/or dot plot of the dose values.$<$br /$>$ & 0.1.7 & 2015-11-03 & 11:40:32
 & $<$br /$>$ Michael Dietze, GFZ Potsdam (Germany),  Sebastian Kreutzer,$<$br /$>$ IRAMAT-CRP2A, Universite Bordeaux Montaigne (France)  Inspired by a plot$<$br /$>$ introduced by Galbraith \& Green (1990)$<$br /$>$  R Luminescence Package Team \\ 
  54 & plot\_DRTResults & Visualise dose recovery test results & $<$br /$>$ The function provides a standardised plot output for dose recovery test$<$br /$>$ measurements.$<$br /$>$ & 0.1.8 & 2015-10-23 & 17:42:47
 & $<$br /$>$ Sebastian Kreutzer, IRAMAT-CRP2A, Universite Bordeaux Montaigne$<$br /$>$ (France), Michael Dietze, GFZ Potsdam (Germany)$<$br /$>$  R Luminescence Package Team \\ 
  55 & plot\_GrowthCurve & Fit and plot a growth curve for luminescence data (Lx/Tx against dose) & $<$br /$>$ A dose response curve is produced for luminescence measurements using a$<$br /$>$ regenerative protocol.$<$br /$>$ & 1.8.1 & 2015-10-31 & 14:56:58
 & $<$br /$>$ Sebastian Kreutzer, IRAMAT-CRP2A, Universite Bordeaux Montaigne$<$br /$>$ (France),   Michael Dietze, GFZ Potsdam (Germany)$<$br /$>$  R Luminescence Package Team \\ 
  56 & plot\_Histogram & Plot a histogram with separate error plot & $<$br /$>$ Function plots a predefined histogram with an accompanying error plot as$<$br /$>$ suggested by Rex Galbraith at the UK LED in Oxford 2010.$<$br /$>$ & 0.4.4 & 2015-09-22 & 09:41:30
 & $<$br /$>$ Michael Dietze, GFZ Potsdam (Germany),   Sebastian Kreutzer,$<$br /$>$ IRAMAT-CRP2A, Universite Bordeaux Montaigne (France)$<$br /$>$  R Luminescence Package Team \\ 
  57 & plot\_KDE & Plot kernel density estimate with statistics & $<$br /$>$ Plot a kernel density estimate of measurement values in combination with the$<$br /$>$ actual values and associated error bars in ascending order. Optionally,$<$br /$>$ statistical measures such as mean, median, standard deviation, standard$<$br /$>$ error and quartile range can be provided visually and numerically.$<$br /$>$ & 3.5 & 2015-09-22 & 09:41:30
 & $<$br /$>$ Michael Dietze, GFZ Potsdam (Germany),  Sebastian Kreutzer,$<$br /$>$ IRAMAT-CRP2A, Universite Bordeaux Montaigne$<$br /$>$  R Luminescence Package Team \\ 
  58 & plot\_NRt & Visualise natural/regenerated signal ratios & $<$br /$>$ This function creates a Natural/Regenerated signal vs. time (NR(t)) plot$<$br /$>$ as shown in Steffen et al. 2009$<$br /$>$ &  &  &  & $<$br /$>$ Christoph Burow, University of Cologne (Germany)$<$br /$>$ \\ 
  59 & plot\_RadialPlot & Function to create a Radial Plot & $<$br /$>$ A Galbraith's radial plot is produced on a logarithmic or a linear scale.$<$br /$>$ & 0.5.3 & 2015-09-22 & 09:41:30
 & $<$br /$>$ Michael Dietze, GFZ Potsdam (Germany),  Sebastian Kreutzer,$<$br /$>$ IRAMAT-CRP2A, Universite Bordeaux Montaigne (France)  Based on a rewritten$<$br /$>$ S script of Rex Galbraith, 2010$<$br /$>$  R Luminescence Package Team \\ 
  60 & plot\_Risoe.BINfileData & Plot single luminescence curves from a BIN file object & $<$br /$>$ Plots single luminescence curves from an object returned by the$<$br /$>$ read\_BIN2R  function.$<$br /$>$ & 0.4.1 & 2015-09-22 & 09:41:30
 & $<$br /$>$ Sebastian Kreutzer, IRAMAT-CRP2A, Universite Bordeaux Montaigne$<$br /$>$ (France),  Michael Dietze, GFZ Potsdam (Germany)$<$br /$>$  R Luminescence Package Team \\ 
  61 & plot\_RLum.Analysis & Plot function for an RLum.Analysis S4 class object & $<$br /$>$ The function provides a standardised plot output for curve data of an$<$br /$>$ RLum.Analysis S4 class object$<$br /$>$ & 0.2.9 & 2015-11-11 & 18:18:13
 & $<$br /$>$ Sebastian Kreutzer, IRAMAT-CRP2A, Universite Bordeaux Montaigne$<$br /$>$ (France)$<$br /$>$  R Luminescence Package Team \\ 
  62 & plot\_RLum.Data.Curve & Plot function for an RLum.Data.Curve S4 class object & $<$br /$>$ The function provides a standardised plot output for curve data of an$<$br /$>$ RLum.Data.Curve S4 class object$<$br /$>$ & 0.1.5 & 2015-09-22 & 09:41:30
 & $<$br /$>$ Sebastian Kreutzer, IRAMAT-CRP2A, Universite Bordeaux Montaigne$<$br /$>$ (France)$<$br /$>$  R Luminescence Package Team \\ 
  63 & plot\_RLum.Data.Image & Plot function for an  RLum.Data.Image  S4 class object & $<$br /$>$ The function provides a standardised plot output for image data of an$<$br /$>$ RLum.Data.Image S4 class object, mainly using the plot functions$<$br /$>$ provided by the  raster  package.$<$br /$>$ & 0.1 & 2015-09-22 & 09:41:30
 & $<$br /$>$ Sebastian Kreutzer, IRAMAT-CRP2A, Universite Bordeaux Montaigne$<$br /$>$ (France)$<$br /$>$  R Luminescence Package Team \\ 
  64 & plot\_RLum.Data.Spectrum & Plot function for an RLum.Data.Spectrum S4 class object & $<$br /$>$ The function provides a standardised plot output for spectrum data of an$<$br /$>$ RLum.Data.Spectrum S4 class object$<$br /$>$ & 0.4.0 & 2015-09-22 & 09:41:30
 & $<$br /$>$ Sebastian Kreutzer, IRAMAT-CRP2A, Universite Bordeaux Montaigne$<$br /$>$ (France)$<$br /$>$  R Luminescence Package Team \\ 
  65 & plot\_RLum & General plot function for RLum S4 class objects & $<$br /$>$ Function calls object specific plot functions for RLum S4 class objects.$<$br /$>$ & 0.4.1 & 2015-09-22 & 09:41:30
 & $<$br /$>$ Sebastian Kreutzer, IRAMAT-CRP2A, Universite Bordeaux Montaigne$<$br /$>$ (France)$<$br /$>$  R Luminescence Package Team \\ 
  66 & plot\_RLum.Results & Plot function for an RLum.Results S4 class object & $<$br /$>$ The function provides a standardised plot output for data of an RLum.Results$<$br /$>$ S4 class object$<$br /$>$ & 0.2.1 & 2015-10-28 & 11:02:52
 & $<$br /$>$ Christoph Burow, University of Cologne (Germany), Sebastian Kreutzer, IRAMAT-CRP2A,$<$br /$>$ Universite Bordeaux Montaigne (France)$<$br /$>$  R Luminescence Package Team \\ 
  67 & plot\_ViolinPlot & Create a violin plot & $<$br /$>$ Draws a kernal densiy plot in combination with a boxplot in its middle. The shape of the violin$<$br /$>$ is constructed using a mirrored density curve. This plot is especially designed for cases$<$br /$>$ where the individual errors are zero or to small to be visualised. The idea for this plot is$<$br /$>$ based on the the 'volcano plot' in the ggplot2 package by Hadely Wickham and Winston Chang.$<$br /$>$ The general idea for the Violin Plot seems to be introduced by Hintze and Nelson (1998).$<$br /$>$ & 0.1.0 & 2015-10-06 & 23:28:52
 & $<$br /$>$ Sebastian Kreutzer, IRAMAT-CRP2A, Universite Bordeaux Montaigne (France)$<$br /$>$  R Luminescence Package Team \\ 
  68 & read\_BIN2R & Import Risoe BIN-file into R & $<$br /$>$ Import a *.bin or a *.binx file produced by a Risoe DA15 and DA20 TL/OSL$<$br /$>$ reader into R.$<$br /$>$ & 0.11.0 & 2015-11-11 & 18:18:13
 & $<$br /$>$ Sebastian Kreutzer, IRAMAT-CRP2A, Universite Bordeaux Montaigne$<$br /$>$ (France), Margret C. Fuchs, HZDR Freiberg, (Germany)$<$br /$>$  R Luminescence Package Team \\ 
  69 & read\_Daybreak2R & Import Daybreak ASCII dato into R & $<$br /$>$ Import a *.txt (ASCII) file produced by a Daybreak reader into R.$<$br /$>$ & 0.2.0 & 2015-10-16 & 18:19:05
 & $<$br /$>$ Sebastian Kreutzer, IRAMAT-CRP2A, Universite Bordeaux Montaigne$<$br /$>$ (France)  Based on suggestion by Willian Amidon and Andrew Louis Gorin.$<$br /$>$  R Luminescence Package Team \\ 
  70 & read\_SPE2R & Import Princeton Intruments (TM) SPE-file into R & $<$br /$>$ Function imports Princeton Instruments (TM) SPE-files into R environment and$<$br /$>$ provides  RLum  objects as output.$<$br /$>$ & 0.1.0 & 2015-09-22 & 09:41:30
 & $<$br /$>$ Sebastian Kreutzer, IRAMAT-CRP2A, Universite Bordeaux Montaigne$<$br /$>$ (France)$<$br /$>$  R Luminescence Package Team \\ 
  71 & read\_XSYG2R & Import XSYG files to R & $<$br /$>$ Imports XSYG files produced by a Freiberg Instrument lexsyg reader into R.$<$br /$>$ & 0.5.2 & 2015-10-19 & 18:38:01
 & $<$br /$>$ Sebastian Kreutzer, IRAMAT-CRP2A, Universite Bordeaux Montaigne$<$br /$>$ (France)$<$br /$>$  R Luminescence Package Team \\ 
  72 & Risoe.BINfileData-class & Class  "Risoe.BINfileData" & $<$br /$>$ S4 class object for luminescence data in R. The object is produced as output$<$br /$>$ of the function  read\_BIN2R .$<$br /$>$ & 0.2.0
 &  &  & $<$br /$>$ Sebastian Kreutzer, IRAMAT-CRP2A, Universite Bordeaux Montaigne$<$br /$>$ (France)$<$br /$>$  R Luminescence Package Team \\ 
  73 & Risoe.BINfileData2RLum.Analysis & Convert Risoe.BINfileData object to an RLum.Analysis object & $<$br /$>$ Converts values from one specific position of a Risoe.BINfileData S4-class$<$br /$>$ object to an RLum.Analysis object.$<$br /$>$ & 0.2.2 & 2015-11-11 & 23:52:21
 & $<$br /$>$ Sebastian Kreutzer, IRAMAT-CRP2A, Universite Bordeaux Montaigne (France)$<$br /$>$  R Luminescence Package Team \\ 
  74 & Risoe.BINfileData2RLum.Data.Curve & Convert an element from a Risoe.BINfileData object to an RLum.Data.Curve$<$br /$>$ object & $<$br /$>$ The function converts one specified single record from a Risoe.BINfileData$<$br /$>$ object to an RLum.Data.Curve object.$<$br /$>$ & 0.2.0 & 2015-11-11 & 23:52:35
 & $<$br /$>$ Sebastian Kreutzer, IRAMAT-CRP2A, Universite Bordeaux Montaigne (France),$<$br /$>$ Christoph Burow, Universtiy of Cologne (Germany)$<$br /$>$  R Luminescence Package Team \\ 
  75 & RLum-class & Class  "RLum" & $<$br /$>$ Abstract class for data in the package Luminescence$<$br /$>$ &  &  &  & $<$br /$>$ Sebastian Kreutzer, IRAMAT-CRP2A, Universite Bordeaux Montaigne (France)$<$br /$>$ \\ 
  76 & RLum.Analysis-class & Class  "RLum.Analysis" & $<$br /$>$ Object class containing analysis data for protocol analysis.$<$br /$>$ &  &  &  & $<$br /$>$ Sebastian Kreutzer, IRAMAT-CRP2A, Universite Bordeaux Montaigne$<$br /$>$ (France)$<$br /$>$ \\ 
  77 & RLum.Data-class & Class  "RLum.Data" & $<$br /$>$ Generalized virtual data class for luminescence data.$<$br /$>$ &  &  &  & $<$br /$>$ Sebastian Kreutzer, IRAMAT-CRP2A, Universite Bordeaux Montaigne (France)$<$br /$>$ \\ 
  78 & RLum.Data.Curve-class & Class  "RLum.Data.Curve" & $<$br /$>$ Class for luminescence curve data.$<$br /$>$ &  &  &  & $<$br /$>$ Sebastian Kreutzer, IRAMAT-CRP2A, Universite Bordeaux Montaigne (France)$<$br /$>$ \\ 
  79 & RLum.Data.Image-class & Class  "RLum.Data.Image" & $<$br /$>$ Class for luminescence image data (TL/OSL/RF).$<$br /$>$ &  &  &  & $<$br /$>$ Sebastian Kreutzer, IRAMAT-CRP2A, Universite Bordeaux Montaigne (France)$<$br /$>$ \\ 
  80 & RLum.Data.Spectrum-class & Class  "RLum.Data.Spectrum" & $<$br /$>$ Class for luminescence spectra data (TL/OSL/RF).$<$br /$>$ &  &  &  & $<$br /$>$ Sebastian Kreutzer, IRAMAT-CRP2A, Universite Bordeaux Montaigne (France)$<$br /$>$ \\ 
  81 & RLum.Results-class & Class  "RLum.Results" & $<$br /$>$ Object class contains results data from functions.$<$br /$>$ &  &  &  & $<$br /$>$ Sebastian Kreutzer, IRAMAT-CRP2A, Universite Bordeaux Montaigne$<$br /$>$ (France)$<$br /$>$ \\ 
  82 & Second2Gray & Converting equivalent dose values from seconds (s) to gray (Gy) & $<$br /$>$ Conversion of absorbed radiation dose in seconds (s) to the SI unit gray$<$br /$>$ (Gy) including error propagation. Normally used for equivalent dose data.$<$br /$>$ & 0.6.0 & 2015-10-02 & 19:04:55
 & $<$br /$>$ Sebastian Kreutzer, IRAMAT-CRP2A, Universite Bordeaux Montaigne$<$br /$>$ (France),  Michael Dietze, GFZ Potsdam (Germany),  Margret C. Fuchs, HZDR,$<$br /$>$ Helmholtz-Institute Freiberg for Resource Technology$<$br /$>$ (Germany)$<$br /$>$  R Luminescence Package Team \\ 
  83 & set\_Risoe.BINfileData & General accessor function for RLum S4 class objects & $<$br /$>$ Function calls object-specific get functions for RisoeBINfileData S4 class objects.$<$br /$>$ & 0.1 & 2015-09-22 & 09:41:30
 & $<$br /$>$ Sebastian Kreutzer, IRAMAT-CRP2A, Universite Bordeaux Montaigne$<$br /$>$ (France)$<$br /$>$  R Luminescence Package Team \\ 
  84 & set\_RLum & General accessor function for RLum S4 class objects & $<$br /$>$ Function calls object-specific set functions for RLum S4 class objects.$<$br /$>$ & 0.1 & 2015-09-22 & 09:41:30
 & $<$br /$>$ Sebastian Kreutzer, IRAMAT-CRP2A, Universite Bordeaux Montaigne$<$br /$>$ (France)$<$br /$>$  R Luminescence Package Team \\ 
  85 & sTeve & sTeve - sophisticated tool for efficient data validation and evaluation & $<$br /$>$ This function provides a sophisticated routine for comprehensive$<$br /$>$ luminescence dating data analysis.$<$br /$>$ &  &  &  & $<$br /$>$ R Luminescence Team, 2012-2013$<$br /$>$ \\ 
  86 & structure\_RLum & General structure function for RLum S4 class objects & $<$br /$>$ Function calls object-specific get functions for RLum S4 class objects.$<$br /$>$ & 0.1.0 & 2015-10-31 & 22:27:15
 & $<$br /$>$ Sebastian Kreutzer, IRAMAT-CRP2A, Universite Bordeaux Montaigne$<$br /$>$ (France)$<$br /$>$  R Luminescence Package Team \\ 
  87 & template\_DRAC & Create a DRAC input data template (v1.1) & $<$br /$>$ This function returns a DRAC input template (v1.1) to be used in conjunction$<$br /$>$ with the use\_DRAC() function$<$br /$>$ &  &  &  & $<$br /$>$ Christoph Burow, University of Cologne (Germany)$<$br /$>$ \\ 
  88 & tune\_Data & Tune data for experimental purpose & $<$br /$>$ The error can be reduced and sample size increased for specific purpose.$<$br /$>$ & 0.5.0 & 2015-09-24 & 16:32:25
 & $<$br /$>$ Michael Dietze, GFZ Potsdam (Germany)$<$br /$>$  R Luminescence Package Team \\ 
  89 & use\_DRAC & Use DRAC to calculate dose rate data & $<$br /$>$ The function provides an interface from R to DRAC. An R-object or a$<$br /$>$ pre-formatted XLS/XLSX file is passed to the DRAC website and the$<$br /$>$ results are re-imported into R.$<$br /$>$ & 0.1.0 & 2015-10-06 & 23:28:52
 & $<$br /$>$ Sebastian Kreutzer, IRAMAT-CRP2A, Universite Bordeaux Montaigne (France), Michael Dietze,$<$br /$>$ GFZ Potsdam (Germany), Christoph Burow, University of Cologne (Germany) $<$br /$>$  R Luminescence Package Team \\ 
  90 & write\_R2BIN & Export Risoe.BINfileData into Risoe BIN-file & $<$br /$>$ Exports a Risoe.BINfileData object in a *.bin or *.binx file that can be$<$br /$>$ opened by the Analyst software or other Risoe software.$<$br /$>$ & 0.3.2 & 2015-10-06 & 23:28:25
 & $<$br /$>$ Sebastian Kreutzer, IRAMAT-CRP2A, Universite Bordeaux Montaigne$<$br /$>$ (France)$<$br /$>$  R Luminescence Package Team \\ 
   \hline
\end{tabular}
\end{table}

